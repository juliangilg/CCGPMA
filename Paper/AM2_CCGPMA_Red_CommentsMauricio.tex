
%% bare_jrnl.tex
%% V1.4b
%% 2015/08/26
%% by Michael Shell
%% see http://www.michaelshell.org/
%% for current contact information.
%%
%% This is a skeleton file demonstrating the use of IEEEtran.cls
%% (requires IEEEtran.cls version 1.8b or later) with an IEEE
%% journal paper.
%%
%% Support sites:
%% http://www.michaelshell.org/tex/ieeetran/
%% http://www.ctan.org/pkg/ieeetran
%% and
%% http://www.ieee.org/

%%*************************************************************************
%% Legal Notice:
%% This code is offered as-is without any warranty either expressed or
%% implied; without even the implied warranty of MERCHANTABILITY or
%% FITNESS FOR A PARTICULAR PURPOSE! 
%% User assumes all risk.
%% In no event shall the IEEE or any contributor to this code be liable for
%% any damages or losses, including, but not limited to, incidental,
%% consequential, or any other damages, resulting from the use or misuse
%% of any information contained here.
%%
%% All comments are the opinions of their respective authors and are not
%% necessarily endorsed by the IEEE.
%%
%% This work is distributed under the LaTeX Project Public License (LPPL)
%% ( http://www.latex-project.org/ ) version 1.3, and may be freely used,
%% distributed and modified. A copy of the LPPL, version 1.3, is included
%% in the base LaTeX documentation of all distributions of LaTeX released
%% 2003/12/01 or later.
%% Retain all contribution notices and credits.
%% ** Modified files should be clearly indicated as such, including  **
%% ** renaming them and changing author support contact information. **
%%*************************************************************************


% *** Authors should verify (and, if needed, correct) their LaTeX system  ***
% *** with the testflow diagnostic prior to trusting their LaTeX platform ***
% *** with production work. The IEEE's font choices and paper sizes can   ***
% *** trigger bugs that do not appear when using other class files.       ***                          ***
% The testflow support page is at:
% http://www.michaelshell.org/tex/testflow/



%\documentclass[journal,draft,onecolumn]{IEEEtran}
\documentclass[journal]{IEEEtran}
% *** GRAPHICS RELATED PACKAGES ***
%
\ifCLASSINFOpdf
  % \usepackage[pdftex]{graphicx}
  % declare the path(s) where your graphic files are
  % \graphicspath{{../pdf/}{../jpeg/}}
  % and their extensions so you won't have to specify these with
  % every instance of \includegraphics
  % \DeclareGraphicsExtensions{.pdf,.jpeg,.png}
\else
  % or other class option (dvipsone, dvipdf, if not using dvips). graphicx
  % will default to the driver specified in the system graphics.cfg if no
  % driver is specified.
  % \usepackage[dvips]{graphicx}
  % declare the path(s) where your graphic files are
  % \graphicspath{{../eps/}}
  % and their extensions so you won't have to specify these with
  % every instance of \includegraphics
  % \DeclareGraphicsExtensions{.eps}
\fi

\usepackage{hyperref}
\usepackage{stfloats}



%% The amssymb package provides various useful mathematical symbols
\usepackage{amssymb}
\usepackage{framed,multirow}
\usepackage{url,microtype}
\usepackage{mathrsfs,bm} % Notacion de transformadas
\usepackage{amsmath}
\usepackage{amsfonts,eucal}
\usepackage[normalem]{ulem}

\usepackage{epsfig}
\usepackage{epstopdf}
\usepackage{booktabs}
\usepackage{adjustbox}
\usepackage[]{subfigure}
\usepackage{lineno}

\usepackage{tikz}
\usepackage{pgfplots}
%\usetikzlibrary{shadings}
\pgfplotsset{compat=1.16}
%\usetikzlibrary{arrows,shapes,shadows,calc,decorations.pathreplacing,positioning}
%\usetikzlibrary{spy}
\usepgfplotslibrary{fillbetween}

\usepackage{times}
%\usepackage[titletoc,toc,title]{appendix}
\usepackage{natbib}
\usepackage{float}
\usepackage[ruled,linesnumbered]{algorithm2e}
\usepackage{array}
\usepackage{color}
\usepackage[capitalize]{cleveref}
\usepackage{enumerate,multirow}
% \usepackage{lineno}


%%Defining some useful variables 
\providecommand{\promedd}[2]{\mathbb{E}_{#1}\!$\left\{#2\right\}$}% operador de promedio
\providecommand{\ve}[1]{{\bm{#1}}}%
\providecommand{\tr}[1]{{\operatorname{tr}$\left({#1}\right)$}}
\providecommand{\mat}[1]{{\bm{#1}}} %
\providecommand{\var}[1]{${\operatorname{var}}\left\{#1\right\}$}
\newcommand{\Real}{\mathbb{R}}
\newcommand{\N}{\mathbb{N}}
\newcommand{\Z}{\mathbb{Z}}

\DeclareMathOperator{\subconj}{\negthinspace\subset\negthinspace }
\DeclareMathOperator{\en}{\!\,\in\!\,}
\DeclareMathOperator{\igual}{\!\,=\!\,}
\DeclareMathOperator{\dist}{\operatorname{d}}
\DeclareMathOperator{\xx}{\negthickspace\times\negthickspace}
\providecommand{\s}[1]{\negthinspace#1\negthinspace}%

\renewcommand{\refname}{\normalsize References}
\newcommand{\dif}[1]{\mathrm{d}#1}
\newcommand{\Erf}[1]{\mathrm{erf}\left(#1\right)}
\newcommand{\Cov}[1]{\mathrm{cov}\left(#1\right)}
\newcommand{\sign}[1]{\mathrm{sign}(#1)}
\newcommand{\phivec}{\boldsymbol{\phi}}
\newcommand{\muvec}{\boldsymbol{\mu}}
\newcommand{\wfunc}[1]{\textnormal{w}(#1)}
\newcommand{\ex}[1]{\mathbb{E}[#1]}
\providecommand{\ve}[1]{{\mathbf{#1}}}

\DeclareMathOperator{\vecO}{vec} % vec operator
\providecommand{\mat}[1]{{\mathbf{#1}}}
\DeclareMathOperator{\cov}{cov} % Operator for the covariance
\DeclareMathOperator{\KL}{KL} % Operator for the KL divergence
\newcommand{\boldk}{\mathbf{k}} % kernel or covariance
\newcommand{\boldt}{\mathbf{t}} % kernel or covariance
\newcommand{\boldK}{\mathbf{K}} % kernel or covariance
\newcommand{\boldf}{\mathbf{f}} % outputs without noise
\newcommand{\boldB}{\mathbf{B}} % coregionalization matrix
\newcommand{\boldA}{\mathbf{A}} % matrix of coeffcients a_{qi}^r
\newcommand{\boldc}{\mathbf{c}} % matrix of coeffcients a_{qi}^r
\newcommand{\boldu}{\mathbf{u}} % vector for latent function
\newcommand{\bolda}{\mathbf{a}}
\newcommand{\boldm}{\mathbf{m}}
\newcommand{\boldv}{\mathbf{v}}
\newcommand{\boldV}{\mathbf{V}}
\newcommand{\boldW}{\mathbf{W}}
\newcommand{\boldY}{\mathbf{Y}}
\newcommand{\boldy}{\mathbf{y}}
\newcommand{\boldZ}{\mathbf{Z}}
\newcommand{\boldz}{\mathbf{z}}
\newcommand{\boldAtilde}{\mathbf{\widetilde{A}}}
\newcommand{\eye}{\mathbf{I}}   % identity matrix
\newcommand{\boldI}{\mathbf{I}} % identity matrix
\newcommand{\boldUpsi}{\bm{\Upsilon}}
\newcommand{\boldH}{\mathbf{H}} % The whole set of lag vectors
\newcommand{\inputSpace}{\mathcal{X}} % The input space
\newcommand{\params}{\bm{\theta}} % Parameters of LMC model
\newcommand{\veC}{\textbf{\hspace{-0.001in}:}} % Simplified version of the vec operator (vec = :)
\newcommand{\preci}{\mathbf{P}}% Precision for the Gaussians
\newcommand{\dataset}{{\cal D}} % dataset
\newcommand{\fracpartial}[2]{\frac{\partial #1}{\partial  #2}} % Fraction for partial derivatives
\newcommand{\gauss}{\mathcal{N}} % Gaussian density
\newcommand{\bolds}{\mathbf{s}}


\newcommand\redsout{\bgroup\markoverwith{\textcolor{red}{\rule[0.5ex]{2pt}{0.4pt}}}\ULon}
\providecommand{\roj}[1]{\textcolor{red}{\uwave{#1}}}
\providecommand{\ver}[1]{\textcolor{green}{\uwave{#1}}}
\providecommand{\gc}[1]{\textcolor{darkgray}{\uwave{#1}}}
\providecommand{\am}[1]{\textcolor[rgb]{0.1,0.5,0.0}{\upshape{#1}}}
% Leave a blank line between paragraphs instead of using \\

%Comments
\providecommand{\am}[1]{\textcolor[rgb]{0.1,0.5,0.0}{\upshape{#1}}}

\newcolumntype{C}[1]{>{\centering\let\newline\\\arraybackslash\hspace{0pt}}m{#1}}

\usepackage{pifont}% http://ctan.org/pkg/pifont
\newcommand{\cmark}{\ding{51}}%
\newcommand{\xmark}{\ding{55}}%
\usetikzlibrary{bayesnet}

% correct bad hyphenation here
\hyphenation{op-tical net-works semi-conduc-tor}

\newcommand{\comment}[2]{{\color{blue}#1} {\color{red}#2}}
% \AtBeginEnvironment{appendices}{\crefalias{section}{appendix}}
\begin{document}
%
% paper title
% Titles are generally capitalized except for words such as a, an, and, as,
% at, but, by, for, in, nor, of, on, or, the, to and up, which are usually
% not capitalized unless they are the first or last word of the title.
% Linebreaks \\ can be used within to get better formatting as desired.
% Do not put math or special symbols in the title.
\title{Correlated Chained Gaussian Processes \comment{with Multiple
    Annotators}{for Datasets with Multiple Annotators?}}
%
%
% author names and IEEE memberships
% note positions of commas and nonbreaking spaces ( ~ ) LaTeX will not break
% a structure at a ~ so this keeps an author's name from being broken across
% two lines.
% use \thanks{} to gain access to the first footnote area
% a separate \thanks must be used for each paragraph as LaTeX2e's \thanks
% was not built to handle multiple paragraphs
%

\author{J. Gil-Gonz\'alez,
        J. Giraldo,
        A. \'Alvarez-Meza, A. Orozco-Guti\'errez,  and~M.~A.~\'Alvarez% <-this % stops a space
\thanks{J. Gil-Gonz\'alez and A. Orozco are with the Universidad Tecnol\'ogica de Pereira, Colombia, 660003, e-mail:\{jugil,aaog\}@utp.edu.co}% <-this % stops a space
\thanks{J. Giraldo and M. A.\'Alvarez are with the University of Sheffield, UK. email: \{jjgiraldogutierrez1,mauricio.alvarez\}@sheffield.ac.uk}% <-this % stops a space
\thanks{A. \'Alvarez is with the Universidad Nacional de Colombia sede Manizales, 170001, Colombia. email: amalvarezme@unal.edu.co}}
%\thanks{Manuscript received April 19, 2005; revised August 26, 2015.}}

% note the % following the last \IEEEmembership and also \thanks - 
% these prevent an unwanted space from occurring between the last author name
% and the end of the author line. i.e., if you had this:
% 
% \author{....lastname \thanks{...} \thanks{...} }
%                     ^------------^------------^----Do not want these spaces!
%
% a space would be appended to the last name and could cause every name on that
% line to be shifted left slightly. This is one of those "LaTeX things". For
% instance, "\textbf{A} \textbf{B}" will typeset as "A B" not "AB". To get
% "AB" then you have to do: "\textbf{A}\textbf{B}"
% \thanks is no different in this regard, so shield the last } of each \thanks
% that ends a line with a % and do not let a space in before the next \thanks.
% Spaces after \IEEEmembership other than the last one are OK (and needed) as
% you are supposed to have spaces between the names. For what it is worth,
% this is a minor point as most people would not even notice if the said evil
% space somehow managed to creep in.



% The paper headers
\markboth{Journal of \LaTeX\ Class Files,~Vol.~14, No.~8, August~2015}%
{Shell \MakeLowercase{\textit{et al.}}: Bare Demo of IEEEtran.cls for IEEE Journals}
% The only time the second header will appear is for the odd numbered pages
% after the title page when using the twoside option.
% 
% *** Note that you probably will NOT want to include the author's ***
% *** name in the headers of peer review papers.                   ***
% You can use \ifCLASSOPTIONpeerreview for conditional compilation here if
% you desire.




% If you want to put a publisher's ID mark on the page you can do it like
% this:
%\IEEEpubid{0000--0000/00\$00.00~\copyright~2015 IEEE}
% Remember, if you use this you must call \IEEEpubidadjcol in the second
% column for its text to clear the IEEEpubid mark.



% use for special paper notices
%\IEEEspecialpapernotice{(Invited Paper)}




% make the title area
\maketitle


\linenumbers

% As a general rule, do not put math, special symbols or citations
% in the abstract or keywords.
\begin{abstract}
  The labeling process within a supervised learning task is usually
  carried out by an expert, which provides the ground truth (gold
  standard) for each sample. However, in many real-world applications,
  we typically have access to annotations provided by crowds holding
  different and unknown expertise levels. Learning from crowds intends
  to configure machine learning paradigms in the presence of
  multi-labelers, residing on two key assumptions: the labeler’s
  performance does not depend on the input space, and independence
  among the annotators is imposed. Here, we propose the correlated
  chained Gaussian processes from multiple annotators--(CCGPMA)
  approach, which models each annotator's performance as a function of
  the input space and exploits the correlations among
  experts. Experimental results associated with \comment{regression and
  classification}{classification and regression? since classification
  appears in the main paper?} tasks show that our CCGPMA achieves \comment{suitable
  performances}{me parece que se debe ser mas especifico, porque a
  suitable performance no dice muy bien que tan suitable. Normalmente
  uno dice ``improves XXXX compared to a baseline based on XXX'' o
  ``improves XXX compared to XXXX''} from inconsistent labelers, even if the gold standard
  is not available.
\end{abstract}

% Note that keywords are not normally used for peerreview papers.
\begin{IEEEkeywords}
Multiple annotators, Correlated Chained Gaussian Processes, Classification, Regression.
\end{IEEEkeywords}

% For peer review papers, you can put extra information on the cover
% page as needed:
% \ifCLASSOPTIONpeerreview
% \begin{center} \bfseries EDICS Category: 3-BBND \end{center}
% \fi
%
% For peerreview papers, this IEEEtran command inserts a page break and
% creates the second title. It will be ignored for other modes.
\IEEEpeerreviewmaketitle



\section{Introduction}

\IEEEPARstart{S}{upervised} learning requires that a domain expert
labels the instances to built the gold standard (ground
truth)~\cite{zhang2019crowdsourced}. Yet, \comment{the}{remove this} experts are scarce, or
their time is expensive, not mentioning that the labeling task is
tedious and time-consuming~\cite{liu2020truth}. As an alternative, the
labeling is distributed through multiple heterogeneous annotators, who
annotate part of the whole dataset by providing their version of the
hidden ground truth~\cite{kara2015modeling}.  Recently, crowdsourcing
platforms, i.e., Amazon Mechanical Turk--
(AMT)\footnote{https://www.mturk.com/}, have been introduced to
capture labels from multiple sources on large datasets
efficiently. The attractiveness of these platforms lies in that, at a
low cost, it is possible to obtain suitable quality labels. Indeed, in
some cases, such a labeling process can compete with those provided by
experts~\cite{snow2008cheap}. However, in such \comment{an multi-labelers}{multi-labeler}
scenario, each instance is matched with multiple annotations provided
by different sources with unknown and diverse expertise, being
difficult to apply traditional supervised learning
algorithms~\cite{tao2018domain}. In this sense, \emph{learning from
  crowds} has been introduced as a general framework from two main
perspectives: to fit the labels from multiple annotators or to adapt
the supervised learning algorithms~\cite{rizos2020average}.

The first approach is known in the literature as ``label aggregation''
or ``truth inference'', comprising the computation of a single hard
label per sample as an estimation of the ground truth. The
hard labels are then used to feed a standard supervised learning
algorithm~\cite{morales2019scalable}. The straightforward method is the so-called majority voting--(MV), and it has been used in
different multi-labeler problems due to its
simplicity~\cite{zhang2014imbalanced}. Still, MV assumes homogeneity
in annotators' reliability, which is hardly feasible in real
applications, e.g., experts vs. spammers. Furthermore, the consensus
is profoundly impacted by incorrect labels and
outliers~\cite{kara2015modeling}. Conversely, more elaborated models
have been considered to improve the estimation of the correct tag through the well-known Expectation-Maximization--(EM) framework and by facing the imbalanced labeling issue~\cite{dawid1979maximum,zhang2014imbalanced}.

The second approach jointly trains the supervised learning algorithm and models the annotators' behavior. It has been shown that such strategies lead to better performance compared to the ones belonging to label
aggregation. Thus, the features used to train the learning algorithm
provide valuable information to puzzle out the ground
truth~\cite{ruiz2019learning}. The most representative work in this area is exposed in \cite{raykar2010learning}, which offers an EM-based framework to learn the parameters of a logistic regression classifier and model the annotators' behavior by computing their sensitivities and specificities. In fact, such a technique has inspired several models
in the context of \comment{multi-labelers}{multi-labeler} scenarios, including binary
classification~\cite{rodrigues2014gaussian,ruiz2019learning},
multi-class discrimination~\cite{morales2019scalable,gonzalez2015automatic}, regression~\cite{groot2011learning,rodrigues2017learning}, and sequence
labeling~\cite{rodrigues2014sequence}. Furthermore, some works have
addressed the multi-labeler problem using deep learning approaches
\comment{to design an extra layer coding}{typically including an extra
layer that codes} the
annotators' information~\cite{albarqouni2016aggnet,rodrigues2018deep,guan2018said}. 

Two main issues are still unsolved in the context of learning from crowds~\cite{g2019machine}: we need to code the relationships between the input
features and the labelers' performance while revealing relevant annotators' interdependencies. In general,  the annotators' behavior is parametrized through a homogeneous constraint across the input samples. The latter assumption is not correct since an expert makes decisions based not only on his/her expertise but also on the features observed from raw data ~\cite{raykar2010learning}. Besides, it is
widespread to consider independence in the annotators' labels, aiming
to reduce the complexity of the model~\cite{venanzi2014community}, or
based on the fact that it is plausible to guarantee that each labeler
performs the annotation process
individually~\cite{tang2019leveraging}. However, this assumption is
not true since there may exist correlations among the
annotators \cite{zhang2011learning}. For example, if the sources are
humans, the independence assumption is hardly feasible because
knowledge is a social construction; then, people's decisions will be
correlated because they share information or belong to a particular school of
thought~\cite{surowiecki2005wisdom,hahn2018communication}. Now, if we consider that the sources are algorithms, where some of them gather the same math principle, there likely exists a
correlation in their labels~\cite{zhu2019unsupervised}. 

In this work, we propose a probabilistic model, named the
\comment{correlated chained Gaussian Processes for multiple
  annotators--(CCGPMA)}{este tambien puede ser el titulo del paper},
to jointly build a prediction algorithm applicable to
\comment{regression and classification}{de nuevo considerar el orden} tasks. CCGPMA is based on the chained GPs model--(CGP) \cite{saul2016chained}, which is a Multi-GPs framework where the
parameters of an arbitrary likelihood function are modeled with
multiple independent GPs (one GP prior per parameter). Unlike CGP, we consider that multiple correlated GPs model the likelihood's parameters. For doing so, we take as a basis the ideas from a Multi-output GP--(MOGP) regression~\cite{alvarez2012kernels}, where each output is coded as a weighted sum of shared latent functions via a semi-parametric latent factor model--(SLFM)~\cite{teh2005semiparametric}. In contrast to the MOGP, we do not have multiple outputs but multiple functions chained to the given likelihood parameters. From the multiple annotators' point of view, the likelihood parameters are related to the labelers' behavior; thereby, CCGPMA models the labelers' behavior as a function of the input features while also taking into account annotators' interdependencies. Moreover, our proposal is based on the so-called inducing variables framework~\cite{alvarez2010efficient}, in combination with stochastic variational inference~\cite{hoffman2013stochastic}. To the best of our knowledge, this is the first attempt to build a probabilistic approach to model the labelers' behavior as a function of the input features while also considering annotators' interdependencies. Achieved results, using both simulated and real-world data, show how our method can deal with both regression and classification problems from multi-labelers data.

The remainder is organized as follows. Section 2 exposes the related
work and the main contributions of the proposal. Section 3 describes the
methods. Sections 4 and 5 present the experiments and discuss the
results. Finally, Section 6 outlines the conclusions and future work.

\section{Related work and main contributions}

Most of the learning from crowds-based methods aim to model the
annotators' behavior based on the
accuracy~\cite{rodrigues2013learning}, the confusion
matrix~\cite{gonzalez2015automatic}, the error
variance~\cite{raykar2010learning}, and the
bias~\cite{rodrigues2017learning}. Concerning this, the expert
parameters are modeled as fixed points~\cite{rodrigues2014gaussian},
or as random variables, where it is considered that such parameters
are homogeneous across the input data~\cite{morales2019scalable}.

The first attempt to analyze the relationship between the annotators'
parameters and the input features is the work
in~\cite{zhang2011learning}. The authors propose an approach for
binary classification with multiple labelers, where the input data is
represented by a defined cluster using a Gaussian Mixture
Model--(GMM). The approach assumes that the annotators exhibit a
particular performance measured in terms of sensitivity and
specificity for each group. However, the model does not consider the
information from multiple experts as an input for the GMM, yielding
variations in the labelers' parameters. Similarly, in
\cite{yan2014learning}, the authors propose a binary classification
algorithm that employs two probability models to code the annotators'
performance as a function of the input space, namely a Bernoulli and a
Gaussian distribution. The parameters of these distributions are
computed via Logistic regression. Nonetheless, a linear dependence
between the labeler expertise and the input space is assumed, which
may not be appropriate because of the data structure's
nonlinearities. For example, if we consider online annotators
assessing some documents, they may have different labeling
accuracy. Such differences may rely on whether they are more familiar
with \comment{some issues}{cuales issues? esto suena abstracto. Se
  puede ser mas concreto usando el ejemplo de los documentos} than
other~\cite{wang2016bi}. Authors in \cite{xiao2013learning} offer a
GP-based regression with multiple annotators. An additional GP models
the annotators' parameters as a nonlinear function of the input
space. Yet, the inference is carried out based on maximum a posteriori
(MAP), without including \comment{uncertainty parameters}{the
  uncertainty of the posterior distribution}.

On the other hand, it has been shown that the relaxation of the
annotators' independence restriction can improve the ground truth
estimation~\cite{zhang2011learning,g2019machine}. To the best of our
knowledge, only two works address such an issue. First, the authors in
\cite{zhu2019unsupervised} describe an approach to deal with
regression problems, where the labelers' behavior is modeled using a
multivariate Gaussian distribution. Thus, the annotators'
interdependencies are coded in the covariance matrix. Further, in
\cite{gil2018learning}, the authors propose a binary classification
method based on a weighted combination of classifiers. In turn, the
weights are estimated by using a kernel alignment-based algorithm
considering dependencies among the labelers.

Here, we propose a GPs-based framework to face classification and
regression settings with multiple annotators. Our proposal follows the
line of the works
in~\cite{rodrigues2014gaussian,groot2011learning,ruiz2019learning,morales2019scalable,morales2019scalable1}
in the sense that we are modeling the unknown ground truth trough a GP
prior. However, while such approaches code the annotators' parameters
as fixed points \cite{rodrigues2014gaussian,groot2011learning}; or
\comment{as}{as a}
random variable
\cite{ruiz2019learning,morales2019scalable,morales2019scalable1}; we
model them as random processes to take into account dependencies
between the input space and the labelers' behavior. Besides, our
CCGPMA shares some similarities with the works
in~\cite{yan2014learning,xiao2013learning}, because we aim to model
the dependencies between the input features and the labelers'
performance. Our method is also similar to the works
in~\cite{zhu2019unsupervised,gil2018learning}, because they assume
dependencies in the annotators' labels. In contrast, CCGPMA is the
only one that includes both assumptions to code the annotators'
behavior. Of note, we highlight that our proposal codes inconsistent
annotations, being robust against outliers.  Namely, CCGPMA can
estimate the annotators' performance for every region in the input
space; meanwhile, state-of-the-art techniques assess it based on a
conventional
averaging~\cite{rodrigues2017learning,morales2019scalable,ruiz2019learning}. \cref{tab:SOA}
summarizes the key insights of our CCGPMA and state-of-the-art
approaches.


\begin{table*}[!tb]
	\centering
	\caption{Survey of relevant supervised learning models devoted to multiple annotators.}
	\resizebox{1\linewidth}{!}{
		\begin{tabular}{C{5cm}C{4cm}C{2.5cm}C{2cm}C{2cm}C{2cm}}\toprule
			Source & Data type & Type of model & Modeling the annotator’s expertise & Expertise as a function of the input space & Modeling the annotators' interdependencies 
            \\\midrule
            \textit{Raykar et al., 2010} \cite{raykar2010learning} & Regression-Binary-Categorical & Probabilistic & \cmark & \xmark & \xmark\\
            \textit{Zhang and Obradovic, 2011} \cite{zhang2011learning} & Binary & Probabilistic & \cmark & \cmark & \xmark\\
            \textit{Xiao et al., 2013} \cite{xiao2013learning} & Regression & Probabilistic & \cmark & \cmark & \xmark\\
            \textit{Yan et al., 2014} \cite{yan2014learning} & Binary & Probabilistic & \cmark & \cmark & \xmark\\
            \textit{Wang and Bi, 2016} \cite{wang2016bi}& Binary & Deterministic & \cmark & \cmark & \xmark\\
            \textit{Rodrigues et al., 2017} \cite{rodrigues2017learning} & Regression-Binary-Categorical & Probabilistic & \cmark & \xmark & \xmark\\
            \textit{Gil-Gonzalez et al., 2018} \cite{gil2018learning} & Binary & Deterministic & \cmark & \xmark & \cmark\\
            \textit{Hua et al., 2018} \cite{hua2018collaborative} & Binary-Categorical & Deterministic & \cmark & \xmark & \xmark\\
            \textit{Ruiz et al., 2019} \cite{ruiz2019learning} & Binary & Probabilistic & \cmark & \xmark & \xmark\\
            \textit{Morales- ́Alvarez et al., 2019} \cite{morales2019scalable} & Binary & Probabilistic & \cmark & \xmark & \xmark\\
            \textit{Zhu et al., 2019} \cite{zhu2019unsupervised} & Regression & Probabilistic & \cmark & \xmark & \cmark\\
            \textbf{Proposal-(CCGPMA)} & Regression-Binary-Categorical & Probabilistic & \cmark & \cmark & \cmark\\\bottomrule
                \end{tabular}
              } 
	\label{tab:SOA}
\end{table*}



\section{Methods}\label{sec:methods}

\subsection{Correlated chained Gaussian processes}

Let us consider an input-output dataset $\dataset\igual\{\ve{X}\s{\in}\mathcal{X},\ve{y}\s{\in}\mathcal{Y}\},$ where $\mat{X}\igual \left\{\ve{x}_n\s{\in}\mathcal{X}\!\subseteq\!\Real^P\right\}_{n=1}^{N}$ and $\mat{y}\igual\left\{y_n\s{\in} \mathcal{Y}\right\}_{n=1}^{N}$. In turn, let a GP be a collection of random variables $f(\ve{x})$ indexed by the input samples $\ve{x}\en \mathcal{X}$ holding a joint multivariate Gaussian distribution~\cite{rasmussen2006gaussian}. A GP is defined by its mean $m(\ve{x})\igual \ex{f(\ve{x})}$ (we consider $m(\ve{x})\igual0$) and covariance function $\kappa_f(\ve{x}, \ve{x^{\prime}})\igual \ex{(f(\ve{x})-m(\ve{x}))(f(\ve{x'})-m(\ve{x'}))}$, where $\kappa_f\s{:}\mathcal{X}\times\mathcal{X}\s{\to}\Real$ is a given kernel function and $\ve{x}^\prime\en\mathcal{X}$, yielding:

\begin{align} f(\ve{x}) \sim \mathcal{GP} (0, \kappa_f(\ve{x}, \ve{x^{\prime}})).
\end{align}
If we consider the finite set of inputs in $\mat{X}$, then $\ve{f}\igual\left[f(\ve{x}_1), \dots , f(\ve{x}_N)\right]^{\top}\en \Real^{N}$ is drawn for a multivariate Gaussian distribution $\ve{f} \sim \gauss(\ve{f}|\ve{0},\mat{K}_{\ve{f}\ve{f}})$, where $\mat{K}_{\ve{f}\ve{f}}\en\Real^{N\times N}$ is the covariance matrix formed by the evaluation of $\kappa_f(\cdot,\cdot)$ over the input set $\mat{X}$.
Accordingly, using GPs for modeling the input-output data collection $\dataset$ consists of constructing a joint distribution between a given likelihood function and one or multiple GP-based priors. To code each likelihood parameter as a random process, we employ the so-called chained GP--(CGP) that attaches such parameters to multiple independent GP priors, as follows~\cite{saul2016chained}:

\begin{align}
    \notag p(\ve{y}, \hat{\ve{f}}| \mat{X}) =& \prod_{n=1}^{N}p(y_n|\theta_1(\ve{x}_n),\dots , \theta_J(\ve{x}_n))\times \cdots \\
    &\cdots \times \prod_{j=1}^{J}\gauss(\ve{f}_j|\ve{0},\mat{K}_{\ve{f}_j\ve{f}_j})\label{eq:CGP},
\end{align}
where each $\{\theta_j(\ve{x}) \en \mathcal{M}_j\}_{j=1}^{J}$ represents the likelihood's parameters, being $J\en\mathbb{N}$ the number of parameters to represent the likelihood. Besides, each $\theta_j(\ve{x})$ holds a non-linear mapping from a GP prior, e.g., $\theta_j(\ve{x})\igual h_j(f_j(\ve{x}))$, where $h_j\s{:}\Real\to \mathcal{M}_j$ is a deterministic function that maps each latent function--(LF) $f_j(\ve{x}),$ to the appropriate domain $\mathcal{M}_j$. Moreover, $\ve{f}_j\igual\left[f_j(\ve{x}_1), \dots , f_j(\ve{x}_N)\right]^{\top}\en\Real^{N}$ is a LF vector that follows a GP prior, and $\hat{\ve{f}}\igual\left[\ve{f}_1, \dots , \ve{f}_J\right]^{\top}\en \Real^{NJ}$. $\mat{K}_{\ve{f}_j\ve{f}_j} \en\Real^{N\times N}$ is the covariance matrix belonging to the $j$-th GP prior, which is computed based on the kernel function  $\kappa_{j}\s{:}\mathcal{X}\times\mathcal{X}\s{\to}\Real$.

From the above, we note that the CGP model assumes independence
between priors, thereby lacking a correlation structure between
GPs. \comment{The latter is unrealistic because one would expect the
  parameters to be correlated}{suena un poco fuerte y no creo que
  siempre sea cierto. Mejor algo como: ``As mentioned before, we 
  consider that the annnotators are correlated. We will enable this
  aspect of the model by
  assuming dependencies among the latent parameters of the chained
  GP''}. \comment{Therefore}{In particular}, we introduce the correlated chained GPs--(CCGP) to model correlations between the GP latent functions, which are supposed to be generated from a semi-parametric latent factor model--(SLFM)~\cite{teh2005semiparametric}:

\begin{align}
\label{eq:SLFM}
f_j(\ve{x}_n) = \sum_{q=1}^{Q} w_{j,q} \mu_{q}(\ve{x}_n),
\end{align}
where $f_j\s{:}\mathcal{X}\s{\to}\Real$ is an LF,
\comment{${k}_q\s{:}\mathcal{X}{\times}\mathcal{X}\s{\to}\Real$ is a
  kernel function, $\mu_q(\cdot) \sim \mathcal{GP}(0,{k}_q(\cdot,
  \cdot))$}{antes de definir el kernel, debe definir u}, and $w_{j,q}\en\Real$ is a combination coefficient ($Q\en\mathbb{N}$). Here, each LF is chained to the likelihood's parameters to extend the joint distribution in \cref{eq:CGP} as follows:

\begin{align}
p(\ve{y},\hat{\ve{f}},\ve{u}|\mat{X})=p(\ve{y}|\bm{\theta}) \prod_{j=1}^{J}p(\ve{f}_j|{\ve{u}})p({\ve{u}}),
\end{align} 
where $\bm{\theta}\igual[\bm{\theta}_1,\dots ,\bm{\theta}_J]^{\top}\en\Real^{NJ}$ holds the model's parameters and $\bm{\theta}_j\igual[\theta_j(\ve{x}_1),\dots ,\theta_j(\ve{x}_N)]^{\top}\en\Real^{N}$ relates the $j$-th parameter with the input space. Our CCGP employs the inducing variables-based method for sparse approximations of GPs~\cite{hensman2015scalable}. For each $\mu_q(\cdot)$, we introduce a set of $M\s{\leq}N$ ``pseudo variables'' $\ve{u}_ q\igual [\mu_q(\ve{z}_1^q), \dots , \mu_q(\ve{z}_M^q)]^{\top}\en \Real^{M}$ through evaluations of $\mu_q(\cdot)$ at unknown locations $\mat{Z}_q \igual [\ve{z}_1^q, \dots , \ve{z}_M^q]\en \Real^{M\times P}$. Note that ${\ve{u}} \igual \left[\ve{u}_1^{\top}, \dots , \ve{u}_Q^{\top} \right]^{\top} \en \Real^{QM}$, yielding:

\begin{align}
\label{eq:CCGPprior}
p(\ve{f}_j|{\ve{u}})=& \gauss\left(\ve{f}_j|\mat{K}_{\ve{f}_j{\ve{u}}}\mat{K}^{-1}_{{\ve{u}}{\ve{u}}}{\ve{u}},\mat{K}_{\ve{f}_j\ve{f}_j}\right.-\cdots \nonumber\\
&\cdots-\left.\mat{K}_{\ve{f}_j{\ve{u}}}\mat{K}^{-1}_{{\ve{u}}{\ve{u}}}\mat{K}_{{\ve{u}}\ve{f}_j}\right),\\
p({\ve{u}}) =& \gauss\left({\ve{u}}| \ve{0}, \mat{K}_{{\ve{u}}{\ve{u}}}\right)\igual \prod_{q=1}^{Q}\gauss(\ve{u}_q|\ve{0},\mat{K}_{\ve{u}_q\ve{u}_q}),
\end{align}
where $\mat{K}_{{\ve{u}}{\ve{u}}} \en \Real^{QM\times QM}$ is a block-diagonal matrix with blocks $\mat{K}_{\ve{u}_q\ve{u}_q}\en \Real^{M\times M}$, based on the kernel function ${\kappa}_q(\cdot,\cdot)$.  The covariance matrix $\mat{K}_{\ve{f}_j\ve{f}_j}\en \Real^{N\times N}$ 
holds elements
$\sum_{q=1}^{Q}w_{j,q}w_{j,q}{\kappa}_q(\ve{x}_n,\ve{x}_{n'})$, with
$\ve{x}_n,\ve{x}_{n'}\en\mat{X}$. Likewise,
$\mat{K}_{\ve{f}_j{\ve{u}}}\igual [\mat{K}_{\ve{f}_j\ve{u}_1}, \dots ,
\mat{K}_{\ve{f}_j\ve{u}_Q}]\en \Real^{N\times QM},$ where
$\mat{K}_{\ve{f}_j\ve{u}_q} \en \Real^{N\times M}$ gathers elements
$w_{j,q}{\kappa}_q(\ve{x}_n, \ve{z}^q_m),$ $m\en\{1,\dots,M\}.$
\comment{It is worth  mentioning  that the inducing variables-based
  scheme reduces the computational complexity in GPs-based
  frameworks~\cite{alvarez2010efficient}}{a menos que demos la
  complejidad computacional exacta, me parece que se puede quitar esta
sentencia}.
\comment{Since in most cases computing the posterior
  distribution}{creo que esto debe ser mas concreto. ``In general, we
  consider that computing the posterior distribution''}
$p(\hat{\ve{f}},\ve{u}|\ve{y})$ is not possible in closed form,
\comment{}{so} we resort to a deterministic approximation of the posterior distribution using variational inference. Therefore, the actual posterior can be approximated by a parametrized variational distribution $p(\hat{\ve{f}},{\ve{u}}|\mat{Y})\s{\approx} q(\hat{\ve{f}},{\ve{u}})$, as:

\begin{align}
\label{eq:VarCCGP}
q(\hat{\ve{f}}, {\ve{u}}) = p(\hat{\ve{f}}|{\ve{u}})q({\ve{u}})= \prod_{j=1}^{J}p(\ve{f}_j|{\ve{u}})\prod_{q=1}^{Q}q(\ve{u}_q),
\end{align}
where $p(\ve{f}_j|{\ve{u}})$ is given by \cref{eq:CCGPprior}, $q(\ve{u}_q)\igual\gauss(\ve{u}_q|\ve{m}_q,\mat{V}_q)$, and $q({\ve{u}})\igual \gauss({\ve{u}}|\ve{m},\mat{V})$. Also, $\ve{m}_q\en\Real^{M}$, and $\mat{V}_q\en \Real^{M\times M}$ are respectively the mean and covariance of variational distribution $q(\ve{u}_q)$; similarly, $\ve{m} \igual [\ve{m}_1^{\top}, \dots , \ve{m}_Q^{\top}]^{\top}\en \Real^{QM}$, and $\mat{V}\en \Real^{QM\times QM}$ is a block-diagonal matrix with blocks given by the covariance matrices $\mat{V}_q$. We remark that the variational approximation given by \cref{eq:VarCCGP} is not uncommon, and it has been used in several GPs models, including \cite{saul2016chained,moreno2018heterogeneous}.
The approximation for the posterior distribution comprises the computation of the following variational parameters: the mean vectors $\{\ve{m}_q\}_{q=1}^{Q}$ and the covariance matrices $\{\mat{V}_q\}_{q=1}^{Q}$. Such an estimation is carried out by maximizing an evidence lower bound--(ELBO), which is given as:

\begin{align}
\mathcal{L}
=&\sum_{n=1}^{N}\mathbb{E}_{\prod\limits^J_{j=1}q(\ve{f}_1),\dots , q(\ve{f}_J)}\left[\log p(y_n|\theta_1(\ve{x}_n),\dots , \theta_J(\ve{x}_n))\right]-\cdots\nonumber\\
&\cdots-\sum_{q=1}^{Q} \mathbb{D}_{KL}(q(\ve{u}_q)||p(\ve{u}_q))\label{eq:LowBound21},
\end{align}
where $\mathbb{D}_{KL}(\cdot||\cdot)$ is the Kullback-Leibler divergence and $q(\ve{f}_j)$ is defined as follows:

\begin{align}
\notag q(\ve{f}_j) &=\gauss(\ve{f}_j|\mat{K}_{\ve{f}_j\ve{u}}\mat{K}_{\ve{u}\ve{u}}^{-1}\ve{m}, \mat{K}_{\ve{f}_j\ve{f}_j}+\cdots\\
& \cdots + \mat{K}_{\ve{f}_j\ve{u}}\mat{K}_{\ve{u}\ve{u}}^{-1}(\mat{V}-\mat{K}_{\ve{u}\ve{u}})\mat{K}_{\ve{u}\ve{u}}^{-1}\mat{K}_{\ve{u}\ve{f}_j}).
\end{align}
Yet, in presence of non-Gaussian likelihoods, the computation of the variational expectations--(VEs) in \cref{eq:LowBound21} cannot be solved analytically~\cite{saul2016chained,moreno2018heterogeneous}. Hence, aiming to model different data types, i.e., classification and regression tasks, we need to find a generic alternative to solve the integrals related to these expectations. In that sense, we use the Gaussian-Hermite quadratures approach as in~\cite{hensman2015scalable,saul2016chained}.\\
It is worth mentioning that the CCGPs objective functions exhibit an
ELBO\comment{W}{?} that allows Stochastic Variational
Inference--(SVI)~\cite{blei2017variational}. Hence, the optimization
is solved through a \textit{mini-batch}-based approach from noisy
estimates of the global objective gradient, which allows dealing with
large scale
datasets~\cite{hensman2015scalable,saul2016chained,moreno2018heterogeneous}. We
remark such ELBO is used to infer the model's hyperparameters such as
the inducing points, the kernel hyperparameters, and the combination
factors $w_{j,q}$ \cref{eq:SLFM}.

\comment{}{En el apendice A, ud incluye algunas ecuaciones para el
  ELBO, pero no se cual de los ELBOs es. Se podrian usar las mismas
  ecuaciones del apendice A para el ELBO que se estudia en esta seccion?}

\subsection{Correlated chained GP for multiple annotators-CCGPMA}

A supervised learning scenario involves the estimation of a function
$g\s{:}\mathcal{X}\s{\to}\mathcal{Y}$. Commonly, each $\ve{x}_n$ is
assigned to a single $y_n$, i.e., the ground truth. \comment{Still, in several
real-world problems instead of the ground truth, we have multiple
labels provided by $R \en\mathbb{N}$ annotators with different levels
of expertise~\cite{raykar2010learning}, where it is common to find
that the $r$-th annotator only labels $|N_r|\s{\le}N$ samples, being
$|N_r|$ the cardinality of the set $N_r\subseteq\{1,\dots , N\}$ that
contains the indexes of samples labeled by the $r$-th
annotator. }{esta sentencia esta muy larga, me parece que se puede
dividir en dos sentencias mas cortas} Besides, we define the set
$R_n\s{\subseteq}\{1, \dots, R\}$ holding the indexes of annotators
that labeled the $n$-th instance. Thereby, the input-output set is
coupled within a multiple annotators scenario as
$\mathcal{D} \igual \left\{\mat{X}, \mat{Y} \igual \{y_n^r\}_{n\en
    N,r\en R_n} \right\}$, where $y_n^r\en \mathcal{Y}$ is the output
given by labeler $r$ to the sample $n$. For specific exposition, in
the following, we focus on categorical labels, i.e.,
$\mathcal{Y}\igual\{1,\dots, K\}$, being $K$ the number of classes
(For the regression case see \cref{CCGPMAReg}).

To model categorical data from multiple annotators using our CCGPMA,
we use the framework proposed in~\cite{rodrigues2013learning}, which
introduces a binary variable $\lambda_n^r\en \{0,1\}$ representing the
$r$-th labeler's reliability as a function of each sample
$\ve{x}_n$. If $\lambda_n^r=1$, the $r$-th annotator is supposed to
provide the actual label, yielding to a categorical
distribution. Conversely, $\lambda_n^r=0$ indicates that the $r$-th
annotator gives an incorrect output, which is modeled by a uniform
distribution. Therefore, the likelihood function is given as:

\begin{align}
\label{eq:ClasLik}
 p(\mat{Y}|\bm{\theta}) &= \prod^N_{n=1}\prod_{r\in R_n}\left(\prod_{k=1}^{K}\zeta_{k,n}^{\delta(y_n^r,k)}\right)^{{\lambda}_n^r}\left(\frac{1}{K}\right)^{(1-{\lambda}_n^r)},
\end{align}
where $\delta(y_n^r,k)\igual 1$, if $y_n^r\igual k$, otherwise $\delta(y_n^r,k)\igual0$. Besides, $\zeta_{k,n}\igual p(y_n^r=k|\lambda_n^r=1)$ is an estimation of the unknown ground truth. Accordingly, $J\igual K+R$ LFs are required within our CCGPMA approach, aiming to model the likelihood' parameters $\bm{\theta}$. In particular, $K$ LFs are used to model $\zeta_{k,n}$ based on a softmax function as:

\begin{align}
\zeta_{k,n} = \frac{\exp(f_k(\ve{x}_n))}{\sum_{j=1}^{K}\exp(f_j(\ve{x}_n))}.
\end{align}
Besides, $R$ LFs are utilized to compute each ${\lambda}_n^r$ from a
step function; therefore, ${\lambda}_n^r\igual 1$ if
$f_{l_r}(\ve{x}_n)\geq 0$, otherwise, ${\lambda}_n^r\igual 0$  ($r\en
\left\{1, \dots R\right\}$). $l_r \igual K+r \en \left\{K+1, \dots
  J\right\}$ indexes the $r$-th annotator' LF. Of note, we approximate
the step function through the well-known sigmoid function to avoid
discontinuities and favor the CCGPMA implementation. \comment{An
  alternative to this formulation was developed}{me parece que se
  deben dar mas detalles de la alternativa. Me imagino que decir que
  los lamdas se modelan usando GP independientes, en lugar de hacer
  parte del SLFM}; however, it was not implemented due to it generates some expensive computations. For more details consult the supplementary material.
%For the regression scenario, see  \cref{CCGPMAReg}.

\comment{}{me parece que hace falta terminar esta seccion describiendo
en un parrafo como se hace la inferencia variacional, asi sea para
decir que la inferencia sigue la misma forma usada en el CCGP}

\section{Experimental Set-up}\label{sec:expsetup}
In this section, we describe the experiments' configurations
to validate our CCGPMA concerning multiple annotators classification
tasks. \comment{}{We also studied regression problems, but due to
  space restrictions the results are included in the supplmental material}

\subsection{Datasets and simulated/provided annotations}\label{sec:datasets}
We test our approach using three types of datasets:  \textit{fully synthetic data}, \textit{semi-synthetic data}, and \textit{fully real datasets}. 

First, we generate \textit{fully synthetic data} as one-dimensional ($P\igual1$) multi-class classification problem ($K\igual3$). The input feature matrix $\mat{X}$ is built by randomly sampling $N\igual100$ points from an uniform distribution within the interval $[0,1]$. The true label for the $n$-th sample is generated by taking the $\arg \max_i \{t_{n,i}\s{:}i\en\{1,2,3\}\}$, where $t_{n,1}\igual\sin(2\pi{x}_n)$, $t_{n,2}\igual-\sin(2\pi{x}_n)$, and $t_{n,3}\igual-\sin(2\pi({x}_n+0.25))+0.5$. Besides, the test instances are obtained by extracting $200$ equally spaced samples from the interval $[0,1]$.

Second, to control the label generation, we build \textit{semi-synthetic data} from seven datasets of the {UCI repository}\footnote{http://archive.ics.uci.edu/ml} focused on binary and multi class-classification: {Wisconsin Breast Cancer Database}--(breast), {BUPA liver disorders}--(bupa), {Johns Hopkins University Ionosphere database}--(ionosphere), {Pima Indians Diabetes Database}--(pima), {Tic-Tac-Toe Endgame database}--(tic-tac-toe), {Wine Data set}--(Wine), and {Image Segmentation Data Set}--(Segmentation). Also, we test the publicly available bearing data collected by the Case Western Reserve University--(Western). The aim is to build a system to diagnose an electric motor's status based on two accelerometers. The feature extraction was performed as in~\cite{hernandez2020bearing}.

Third, we evaluate our proposal on two \textit{fully real datasets}, where both the input features and the annotations are captured from real-world problems. Namely, we use a bio-signal database, where the goal is to build a system to evaluate the presence/absence of voice pathologies. In particular, a subset ($N\igual218$) of the Massachusetts Eye and Ear Infirmary Disordered Voice Database from the Kay Elemetrics company is utilized, which comprises voice records from healthy and different voice issues.  Each signal is parametrized by the Mel-frequency cepstral coefficients (MFCC) to obtain an input space with $P\igual13$. A set of physicians assess the voice quality by following the GRBAS protocol that comprises the evaluation of five qualitative scales: Grade of dysphonia--(G), Roughness--(R), Breathiness--(B), Asthenia--(A), and Strain--(S). For each perceptual scale, the specialist assigns a tag ranging from 0 (healthy voice) to 3 (severe disease)~\cite{arias2011automatic}. Accordingly, we face five multi-class classification problems (one per scale).  We follow the procedure in~\cite{gil2018learning} to rewrite five binary classification tasks preserving the available ground truth~\cite{gonzalez2015automatic}. Further, we use the music genre data\footnote{http://fprodrigues.com/publications/learning-from-multiple-annotators-distinguishing-good-from-random-labelers/}, holding a collection of songs records labeled from one to ten depending on their music genre: classical, country, disco, hip-hop, jazz, rock, blues, reggae, pop, and metal. From this set, $700$ samples were published randomly in the AMT platform to obtain labels from multiples sources (2946 annotations from 44 workers).Yet, we only consider the annotators who labeled at least 20\% of the instances; thus, we use the information from $R \igual 7$ labelers. The feature extraction is performed by following the work by authors in \cite{rodrigues2013learning}, to obtain an input space with $P\igual124$. \cref{tab:ClaData} summarizes the tested datasets for the classification case.

\begin{table}[!tb]
	\caption{Tested datasets.
	}
	\label{tab:ClaData}
	\centering
	\resizebox{\linewidth}{!}{
		\begin{tabular}{ccccC{2cm}cC{2cm}cC{2cm}}\toprule
			&& Name && Number of features && Number of instances && Number of classes\\\midrule
		\multirow{ 1}{*}{\textit{fully synthetic}}	&& synthetic && 1 && 100 && 3\\\midrule
		\multirow{ 8}{*}{\textit{semi-synthetic}}   && Breast&& 9 && 683 && 2\\ 
		                                            && Bupa && 6 && 345 && 2\\
		                                            && Ionosphere && 34 && 351 && 2\\
		                                            && Pima && 8 && 768 && 2\\
		                                            && Tic-tac-toe && 9 && 958 && 2\\
		                                            && Western && 7 && 3413 && 4\\
		                                            && Wine && 13 && 178 && 3\\
		                                            && Segmentation && 18 && 2310 && 7 \\\midrule
      	\multirow{ 2}{*}{\textit{fully real}}       && Voice && 13 && 218 && 2\\
      	                                            && Music && 124 && 1000 && 10\\\bottomrule
	\end{tabular}}
\end{table}

Note that the \textit{fully synthetic} and the \textit{semi-synthetic} datasets do not hold real annotations. Therefore, it is necessary to simulate those labels as corrupted versions of the hidden ground truth. Here, the simulations are performed by assuming: i) dependencies among annotators, and ii) the labelers' performance is modeled as a function of the input features. In turn, an SLFM-based approach is used to build the labels, as follows:

\begin{itemize}
    \item [--] Define $Q$ deterministic functions $\hat{\mu}_q\s{:}\mathcal{X}\to\Real$, and their combination parameters $\hat{w}_{l_r,q}\en \Real$, $\forall r\en R, n\en N$.
    \item[--] Compute $\hat{f}_{l_r,n}\igual \sum_{q=1}^{Q}\hat{w}_{l_r,q}\hat{\mu}_q(\hat{x}_n)$, where $\hat{x}_n\en\Real$ is the $n$-th component of $\hat{\ve{x}}\en \Real^N$, being $\hat{\ve{x}}$ the $1-$D representation of the input features in $\mat{X}$ by using the well-known $t$-distributed Stochastic Neighbor Embedding approach~\cite{maaten2008visualizing}.
    \item[--] Calculate $\hat{\lambda}_{n}^r = \sigma(\hat{f}_{l_r,n})$, where $\sigma(\cdot)\en[0,1]$ is the sigmoid function.
    \item[--] Finally, find the $r$-th label as $y_n^r \igual \begin{cases}y_n, &\mbox{if }\lambda_{n}^r \ge 0.5\\ \tilde{y}_n, & \mbox{if }\lambda_{n}^r <0.5 \end{cases}$, where $\tilde{y}_n$ is a flipped version of the actual label $y_n$.
\end{itemize}

\subsection{Method comparison and performance metrics}

The classification performance is assessed as the Area Under the Curve--(AUC). Further, the AUC is extended for multi-class settings, as discussed by authors in~\cite{fawcett2006introduction}. We use a cross-validation scheme with 15 repetitions where $70$\% of the samples are utilized for training and the remaining $30\%$ for testing (except for the music dataset training and testing sets are clearly defined). \cref{tab:ClaVal} displays the employed methods of the state-of-the-art for comparison purposes. The abreviations are fixed as: Gaussian Processes classifier (GPC), logistic regression classifier (LRC), majority voting (MV), multiple annotators (MA), Modelling annotators expertise (MAE), Learning from crowds (LFC), Distinguishing good from random labelers (DGRL), kernel alignment-based annotator relevance analysis (KAAR).

\begin{table}[bt!]
	\caption{A brief overview of the state-of-the-art methods tested. 
	}
	\label{tab:ClaVal}
	\centering
	\resizebox{1\linewidth}{!}{
		\begin{tabular}{lcl}\toprule
			Algorithm && Description \\\midrule
			GPC-GOLD  && A GPC using the real labels (upper bound).\\
			GPC-MV    &&  A GPC using the MV of the labels as the ground truth.\\
			MA-LFC-C~\cite{raykar2010learning}  && A LRC with constant parameters across the input space.\\
			MA-DGRL~\cite{rodrigues2013learning}   && A multi-labeler approach that considers as latent variables\\ && the annotator performance.\\
            MA-GPC~\cite{rodrigues2014gaussian}	 &&  A multi-labeler GPC, which is as an extension of MA-LFC.\\
            MA-GPCV~\cite{morales2019scalable} &&  An extension of MA-GPC that includes variational inference\\
            &&and priors over the labelers' parameters.\\
            MA-DL~\cite{rodrigues2018deep}  && A Crowd Layer for DL, where the annotators' parameters\\
			&&  are constant across the input space.\\
			KAAR~\cite{gil2018learning}  &&  A kernel-based approach that employs a convex combination\\ 
			 &&  of classifiers and codes labelers dependencies.\\
    		CGPMA-C  &&  A particular case of our CCGPMA for classification,\\
    		&& where $Q\igual J$, and we fix $w_{j,q}\igual 1$, if $j\igual q,$, otherwise $w_{j,q}\igual 0.$\\\bottomrule
	\end{tabular}}
\end{table}

\subsection{CCGPMA training}\label{sec:training}
Overall, the Radial basis function--(RBF) kernel is preferred in pattern classification because of its universal approximating ability and mathematical tractability. Hence, for all GP-based approaches, the kernel functions are fixed as:

\begin{align}\label{eq:RBF}
\kappa(\ve{x}_n, \ve{x}_{n^{\prime}}) = \phi_1\exp\left(\frac{-\|\ve{x}_n- \ve{x}_{n^{\prime}} \|_2^2}{2{\phi}_2^2}\right),
\end{align}
where $\|\cdot\|^2$ stands for the L$2$ norm, $n,n^{\prime} \en \left\{1,2,\dots, N\right\}$, and $\phi_1,\phi_2\en \Real^+$ are the kernel hyper-parameters. For concrete testing, we fix $\phi_1\igual1$, while ${\phi}_2$ is estimated by optimizing the corresponding ELBO (as exposed in~\cref{eq:LowBound21}). Moreover, for CGPMA, we fix $Q\igual R+K$, since each LF $f_j(\cdot)$ is linked to $u_q(\cdot)$. On the other hand, for CCGPMA, each $f_j(\cdot)$ is built as a convex combination of $\mu_q(\cdot)$ (see \cref{eq:SLFM}); therefore, there is no restriction concerning $Q$. However, to make a fair comparison with CGPMA, we also fix $Q\igual R+K$ in CCGPMA. For the \textit{fully synthetic datasets}, we use $M\igual 10$ inducing points per latent function, and for the remaining experiments, we test with $M\igual 40$, and $M\igual 80$.  For all the experiments, we use stochastic inference with a mini-batch size of $100$. The CCGPMA's Python code is publicly available.\footnote{https://github.com/juliangilg/CCGPMA}

\section{Results and Discussion}
% In this section, we expose the results regarding categorical data (classification), for the three types of cases that we have defined in previous section: \textit{fully synthetic data}, \textit{semi synthetic data}, and \textit{fully real data}. Moreover, we expose the results for real-valued data (regression) in the case of \textit{fully real data}; regression results for \textit{semi synthetic data} are in \cref{CCGPMAReg}.

%\subsection{Classification}
\subsection{Fully synthetic data results.} 

We first perform a controlled experiment to test the CCGPMA capability when dealing with binary and multi-class classification. We use the \textit{fully synthetic} dataset described in \cref{sec:datasets}. Besides, five labelers ($R=5$) are simulated with different levels of expertise. To simulate the error-variances, we define $Q\igual3$ $\hat{\mu}_q(\cdot)$ functions, yielding: 

\begin{align}
\label{eq:u1c}
\hat{\mu}_1(x) &= 4.5\cos(2\pi x + 1.5\pi) - 3\sin(4.3\pi x + 0.3\pi),\\
\label{eq:u2c}
\hat{\mu}_2(x) &= 4.5\cos(1.5\pi x + 0.5\pi) + 5\sin(3\pi x + 1.5\pi),\\
\label{eq:u3c}
\hat{\mu}_3(x) &= 1,
\end{align}
where $x\in [0,1]$. Besides, the combination weights are gathered within the following combination matrix $\hat{\mat{W}} \en \Real^{Q\times R}$:

\begin{figure*}[!tb]
	\centering
% 	\begin{tikzpicture}[]
\begin{axis}[scale=.6,name=plot1, xmin=0, xmax=1,ymin=-1.25,ymax=1.1,
x tick label style={font=\footnotesize, align=center},
y tick label style={font=\footnotesize, align=center},
legend style={font=\small},
legend cell align={left},
cycle list name=color list,ytick={-0.8,-0.4,0,0.4,0.8},
%ymajorgrids=true,xmajorgrids=true,grid style=dashed,
title={GPR-GOLD}]
\addplot+[opacity=1,dashed,thick,smooth,black]table{Figures/GT.dat};
\addplot+[opacity=1,blue,very thick,smooth]table{Figures/GPGT.dat};
\addplot+[opacity=0,name path=A,very thick,smooth,red]table{Figures/GPGTU.dat};
\addplot+[opacity=0,name path=B,very thick,smooth,red]table{Figures/GPGTL.dat};
\addplot[opacity=0.3,blue!50] fill between[of=A and B];
\end{axis}
\begin{axis}[scale=.6,name=plot2,at={($(plot1.north east)+(0.12cm,0cm)$)},anchor=north west, xmin=0, xmax=1,ymin=-1.25,ymax=1.1,x tick label style={font=\footnotesize, align=center},
y tick label style={font=\footnotesize, align=center},legend style={font=\small},
legend cell align={left},cycle list name=color list,yticklabel=\empty,xticklabel=\empty, %xticklabel=\empty,ytick={-0.8,-0.4,0,0.4,0.8},ymajorgrids=true,xmajorgrids=true,grid style=dashed,
title={GPR-Av}]
\addplot+[opacity=1,dashed,thick,smooth,black]table{Figures/GT.dat};
\addplot+[opacity=1,blue,very thick,smooth]table{Figures/GPAv.dat};
\addplot+[opacity=0,name path=A,very thick,smooth,red]table{Figures/GPAvU.dat};
\addplot+[opacity=0,name path=B,very thick,smooth,red]table{Figures/GPAvL.dat};
\addplot[opacity=0.3,blue!50] fill between[of=A and B];
\end{axis}
\begin{axis}[scale=.6,name=plot3,at={($(plot2.north east)+(0.12cm,0cm)$)},anchor=north west, xmin=0, xmax=1,ymin=-1.25,ymax=1.1,x tick label style={font=\footnotesize, align=center},
y tick label style={font=\footnotesize, align=center},legend style={font=\small},
legend cell align={left},cycle list name=color list,yticklabel=\empty,xticklabel=\empty, %xticklabel=\empty,ytick={-0.8,-0.4,0,0.4,0.8},ymajorgrids=true,xmajorgrids=true,grid style=dashed,
title={MA-LFCR}]
\addplot+[opacity=1,dashed,thick,smooth,black]table{Figures/GT.dat};
\addplot+[opacity=1,blue,very thick,smooth]table{Figures/LFCR.dat};
\end{axis}
\begin{axis}[scale=.6,name=plot4,at={($(plot3.north east)+(0.12cm,0cm)$)},anchor=north west, xmin=0, xmax=1,ymin=-1.25,ymax=1.1,x tick label style={font=\footnotesize, align=center},
y tick label style={font=\footnotesize, align=center},legend style={font=\small},
legend cell align={left},cycle list name=color list,yticklabel=\empty,xticklabel=\empty, %xticklabel=\empty,ytick={-0.8,-0.4,0,0.4,0.8},ymajorgrids=true,xmajorgrids=true,grid style=dashed,
title={MA-GPR}]
\addplot+[opacity=1,dashed,thick,smooth,black]table{Figures/GT.dat};
\addplot+[opacity=1,blue,very thick,smooth]table{Figures/GPM.dat};
\addplot+[opacity=0,name path=A,very thick,smooth,red]table{Figures/GPU.dat};
\addplot+[opacity=0,name path=B,very thick,smooth,red]table{Figures/GPL.dat};
\addplot[opacity=0.3,blue!50] fill between[of=A and B];
\end{axis}
\begin{axis}[scale=.6,name=plot5,at={($(plot2.south west)+(0cm,-1.0cm)$)},anchor=north west, xmin=0, xmax=1,ymin=-1.25,ymax=1.1,x tick label style={font=\footnotesize, align=center},
y tick label style={font=\footnotesize, align=center},legend style={font=\small},
legend cell align={left},cycle list name=color list,yticklabel=\empty,xticklabel=\empty, %xticklabel=\empty,ytick={-0.8,-0.4,0,0.4,0.8},ymajorgrids=true,xmajorgrids=true,grid style=dashed,
title={MA-DL},cycle list name=color list,legend columns=-1,legend style={fill=white, fill opacity=1, draw opacity=1,text opacity=1},legend style={font=\footnotesize},]
\addplot+[forget plot,opacity=1,dashed,thick,smooth,black]table{Figures/GT.dat};
\addplot+[opacity=1,very thick,smooth,blue]table{Figures/MADLB.dat};
\addlegendentry{B};
\addplot+[opacity=1,very thick,smooth,red]table{Figures/MADLS.dat};
\addlegendentry{S};
\addplot+[opacity=1,very thick,smooth,green]table{Figures/MADLB+S.dat};
\addlegendentry{B+S};
\end{axis}
\begin{axis}[scale=.6,name=plot6,at={($(plot5.north east)+(0.12cm,0cm)$)},anchor=north west, xmin=0, xmax=1,ymin=-1.25,ymax=1.1,x tick label style={font=\footnotesize, align=center},
y tick label style={font=\footnotesize, align=center},legend style={font=\small},
legend cell align={left},cycle list name=color list,yticklabel=\empty, %xticklabel=\empty,ytick={-0.8,-0.4,0,0.4,0.8},ymajorgrids=true,xmajorgrids=true,grid style=dashed,
title={CGPMA-R},xticklabel=\empty,]
\addplot+[opacity=1,dashed,thick,smooth,black]table{Figures/GT.dat};
\addplot+[opacity=1,blue,very thick,smooth]table{Figures/CGPM.dat};
\addplot+[opacity=0,name path=A,very thick,smooth,red]table{Figures/CGPU.dat};
\addplot+[opacity=0,name path=B,very thick,smooth,red]table{Figures/CGPL.dat};
\addplot[opacity=0.3,blue!50] fill between[of=A and B];
\end{axis}
\begin{axis}[scale=.6,name=plot7,at={($(plot6.north east)+(0.12cm,0cm)$)},anchor=north west, xmin=0, xmax=1,ymin=-1.25,ymax=1.1,x tick label style={font=\footnotesize, align=center},
y tick label style={font=\footnotesize, align=center},legend style={font=\small},
legend cell align={left},cycle list name=color list,yticklabel=\empty, xticklabel=\empty,
%ytick={-0.8,-0.4,0,0.4,0.8},ymajorgrids=true,xmajorgrids=true,grid style=dashed,
title={CCGPMA-R}]
\addplot+[opacity=1,dashed,thick,smooth,black]table{Figures/GT.dat};
\addplot+[opacity=1,blue,very thick,smooth]table{Figures/CCGPM.dat};
\addplot+[opacity=0,name path=C,very thick,smooth,blue]table{Figures/CCGPU.dat};
\addplot+[opacity=0,name path=D,very thick,smooth,blue]table{Figures/CCGPL.dat};
\addplot[opacity=0.3,blue!50] fill between[of=C and D];
\end{axis}
\end{tikzpicture}
	\includegraphics[width = 0.90\textwidth]{Figures/SinCla.pdf}
	\caption{Predicted mean label value for the Fully synthetic dataset. We compare the prediction of our CCGPMA-C($AUC=1$) and CCGPMA-C($AUC=0.9999$) algorithms against: the theoretical upper bound GPC-GOLD($AUC=1.0$),  the lower bound GPC-MV($AUC=0.9809$), and the state-of-the-art approaches MA-LFC-C($AUC=0.9993$), MA-DGRL($AUC=0.9999$),  MA-GPC($AUC=0.9977$), MA-GPCV($AUC=0.9515$), MA-DL-MW($AUC=0.9989$), MA-DL-VW($AUC=0.9972$), MA-DL-VW+B($AUC=0.9994$), KAAR($0.9099$). Note that the shaded region in GPC-MV, CGPMA-C, and CCGPMA-C indicates the area enclosed by the mean plus or minus two standard deviations. There is no shaded region for approaches lacking prediction uncertainty.}
	\label{fig:FSCla}
\end{figure*}
\begin{align}
\hat{\mat{W}}=\begin{bmatrix}
0.4  &  0.7   & -0.5 &  0.0  & -0.7\\
0.4   &  -1.0  & -0.1  &  -0.8 & 1.0\\
3.1   &  -1.8  & -0.6  &  -1.2   & 1.0
\end{bmatrix},
\label{eq:parametersPC}
\end{align}
holding elements $\hat{w}_{l_r,q}$. \cref{fig:FSCla} shows the
predictive performance for all approaches for the \textit{fully
  synthetic} data. First, we highlight that the predicted mean label
value--(PMLV) for KAAR, MA-GPC, and MA-GPCV presents a different shape
than the ground truth; moreover, KAAR and MA-GPCV exhibit the worst
AUC, even worse than the intuitive lower bound GPC-MV. We explain such
conduct in the sense that these approaches are designed to deal with
binary labels
~\cite{gil2018learning,rodrigues2014gaussian,ruiz2019learning}. To
face such a problem, we use the \textit{one-vs-all} scheme; still, it
can lead to ambiguously classified
regions~\cite{bishop2006pattern}. We note an akin predictive AUC
concerning MA-DL methods and the linear approaches MA-LFC-C and
MA-DGRL. Nonetheless, the \comment{linear techniques exhibit PMLV}{de
  la forma como se describe en este parrafo, parece que PMLV es una
  metrica de evaluacion, pero no se hablo de ella al inicio de la
  seccion ``Method comparison and performance metrics''}less similar to the Ground truth, which is due to MA-LFC-C and MA-DGRL only can deal with linearly separable data. Further, we analyze the results of our CGPMA-C and its particular enhancement CCGPMA-C. We remark that our methods' predictive AUC is pretty close to deep learning and linear models. Unlike them, our CGPMA-C and CCGPMA-C show the most accurate PMLV compared with the absolute gold standard. CCGPMA-C behaves quite similarly to GPC-GOLD, which is the theoretical upper bound. Finally, from the GPC-MV, we do not identify notable differences with the rest of the approaches (excluding KAAR and MA-GPCV).

\begin{figure*}[!tb]
	\centering
	%\input{Figures/VarEXp.tex}
	\includegraphics[width = 0.8\textwidth]{Figures/VarEXpC.pdf}
	\caption{Fully synthetic data reliability results. From top to bottom, the first column exposes the true reliabilities ($\lambda_r$). The subsequent columns present the estimation of the reliabilities performed by state-of-the-art models, where the correct values are provided in dashed lines. The shaded region in CGPMA-C and CCGPMA-C indicates the area enclosed by the mean $\pm$ two standard deviations.  Also, the accuracy (Acc) is provided.}
	\label{fig:ExpCla}
\end{figure*}
From the above, we recognize that analyzing both the predictive AUC and the PMLV, our CCGPMA-C exhibits the best performance obtaining similar results compared with the intuitive upper bound (GPC-GOLD). Accordingly, CCGPMA-C proffers a more suitable representation of the labelers' behavior than its competitors. Indeed, CCGPMA-C codes both the annotators' dependencies and the relationship between the input features and the annotators' performance. To empirically support the above statement, \cref{fig:ExpCla} shows the estimated per-annotator reliability, where we only take into account models that include such types of parameters (MA-DGRL, CGPMA-C, and CCGPMA-C). As seen, MA-DGRL (see column 2 in \cref{fig:ExpCla}) does not offer a proper representation of the annotators' behavior. CGPMA-C and CCGPMA-C (columns 3 and 4 in \cref{fig:ExpCla}) outperforms MA-DGRL, which is a direct repercussion of modeling the labelers' parameters as functions of the input features. We observe that CCGPMA-C exhibits the best performance in terms of accuracy; such an outcome is due to this method improves the quality of the annotators' model by considering correlations among their decisions~\cite{zhu2019unsupervised,gil2018learning}).

\subsection{Semi-synthetic data results.}

\begin{table*}[!htb]
	\centering
	%\tiny
	\scriptsize 
	\caption{AUC classification results for the semi synthetic datasets. Bold: the highest AUC excluding the upper bound (GPC-GOLD).}
	\resizebox{.98\linewidth}{!}{
	\begin{tabular}{cccccccccc}\toprule
		{Method} & Breast & Bupa & Ionosphere & Pima & TicTacToe & Western & Wine & Segmentation & {Average}\\\midrule
		GPC-GOLD($M=40$)&$0.9907\pm0.0045$ & $0.6975\pm0.0466$ & $0.9490\pm0.0235$ & $0.8378\pm0.0302$ & $0.8429\pm0.0334$ & $0.9185\pm0.0061$ &$0.9987\pm0.0015$& $0.9596\pm0.0196$& $0.8993$\\ 
        GPC-GOLD($M=80$)&$0.9903\pm0.0046$ & $0.6997\pm0.0483$ & $0.9513\pm0.0225$ & $0.8374\pm0.0297$ & $0.8491\pm0.0323$ & $0.9250\pm0.0057$ &$0.9988\pm0.0016$& $0.9781\pm0.0041$& $0.9037$\\ 
        GPC-MV($M=40$)  &$0.9897\pm0.0045$ & $0.5366\pm0.0516$ & $0.7566\pm0.0572$ & $0.5399\pm0.0760$ & $0.6620\pm0.0357$ & $0.8658\pm0.0331$ &$0.8179\pm0.0212$& $0.9562\pm0.0228$& $0.7656$\\ 
        GPC-MV($M=80$)  &$0.9892\pm0.0048$ & $0.5698\pm0.0529$ & $0.7779\pm0.0550$ & $0.5302\pm0.0674$ & $0.6744\pm0.0357$ & $0.8446\pm0.0089$ &$0.8323\pm0.0487$& $0.9749\pm0.0047$& $0.7742$\\ 
        MA-LFC-C        &$0.8789\pm0.0510$ & $0.4593\pm0.1444$ & $0.7358\pm0.0901$ & $0.8119\pm0.0313$ & $0.6004\pm0.0261$ & $0.8400\pm0.0211$ &$0.9692\pm0.0357$& $0.9892\pm0.0031$& $0.7856$\\ 
        MA-DGRL         &$0.9757\pm0.0189$ & $0.5724\pm0.0336$ & $0.6453\pm0.0721$ & $0.8138\pm0.0290$ & $0.6129\pm0.0230$ & $0.8143\pm0.0150$ &$0.9795\pm0.0221$& $0.9897\pm0.0038$& $0.8005$\\ 
        MA-GPC          &$0.9811\pm0.0116$ & $0.5446\pm0.0578$ & $0.6631\pm0.1474$ & $0.5325\pm0.1780$ & $0.6079\pm0.0995$ & $0.8671\pm0.0114$ &$0.9417\pm0.0262$& $0.9734\pm0.0035$& $0.7639$\\ 
        MA-GPCV         &$0.8270\pm0.0547$ & $0.5567\pm0.0683$ & $0.6238\pm0.0871$ & $0.6217\pm0.0590$ & $0.6104\pm0.1003$ & $0.8451\pm0.0147$ &$0.9735\pm0.0172$& $0.9924\pm0.0027$& $0.7563$\\ 
        MA-DL-MW        &$0.9470\pm0.0173$ & $0.5237\pm0.0568$ & $0.7535\pm0.0543$ & $0.6178\pm0.0267$ & $0.6827\pm0.0296$ & $0.9092\pm0.0056$ &$0.9728\pm0.0109$& $0.9950\pm0.0017$& $0.8002$\\ 
        MA-DL-VW        &$0.9526\pm0.0245$ & $0.5327\pm0.0618$ & $0.6987\pm0.0497$ & $0.6063\pm0.0336$ & $0.6771\pm0.0267$ & $0.9173\pm0.0067$ &$0.9807\pm0.0152$& $\mathbf{0.9972\pm0.0011}$& $0.7953$\\ 
        MA-DL-VW+B      &$0.9465\pm0.0242$ & $0.5281\pm0.0631$ & $0.7196\pm0.0453$ & $0.6123\pm0.0378$ & $0.6780\pm0.0342$ & $0.9164\pm0.0085$ &$0.9817\pm0.0155$& $\mathbf{0.9972\pm0.0009}$& $0.7975$\\ 
        KAAR            &$0.8058\pm0.0274$ & $0.5920\pm0.0663$ & $0.7046\pm0.0739$ & $0.5802\pm0.0406$ & $0.6381\pm0.0545$ & $0.8588\pm0.0120$ &$0.9943\pm0.0105$& $0.9217\pm0.0190$& $0.7619$\\ 
        CGPMA-C($M=40$) &$0.9920\pm0.0038$ & $0.5537\pm0.0630$ & $0.8356\pm0.1002$ & $0.8201\pm0.0314$ & $0.7056\pm0.0304$ & $0.9178\pm0.0066$ &$0.9969\pm0.0028$& $0.9679\pm0.0065$& $0.8487$\\ 
        CGPMA-C($M=80$) &$0.9914\pm0.0038$ & $0.5945\pm0.0642$ & $0.8615\pm0.0696$ & $\mathbf{0.8204\pm0.0318}$ & $0.7048\pm0.0312$ & $0.9185\pm0.0057$ &$\mathbf{0.9986\pm0.0016}$& $0.9406\pm0.0061$& $0.8538$\\ 
        CCGPMA-C($M=40$)&$\mathbf{0.9938\pm0.0027}$ & $0.5734\pm0.0533$ & $0.9021\pm0.1079$ & $0.7810\pm0.0622$ & $\mathbf{0.7495\pm0.0539}$ & $0.9269\pm0.0058$ &$0.9952\pm0.0040$& $0.9774\pm0.0048$& $0.8624$\\ 
        CCGPMA-C($M=80$)&$0.9933\pm0.0030$ & $\mathbf{0.6022\pm0.0487}$ & $\mathbf{0.9023\pm0.1066}$ & $0.8045\pm0.0510$ & $0.7312\pm0.0323$ & $\mathbf{0.9307\pm0.0049}$ &$0.9955\pm0.0039$& $0.9774\pm0.0045$& $\mathbf{0.8671}$\\\bottomrule
	\end{tabular}}
	\label{tab:SSClaResults}
\end{table*}
It is worth mentioning that the \comment{dataset}{datasets?} comprise real-world input
features whilst the labels from multiple annotators are
\comment{}{generated} following the
\textit{fully synthetic data} \comment{set-up (see
\cref{eq:u1c,eq:u2c,eq:u3c,eq:parametersPC})}{me imagino que esta
forma de generar las etiquetas es estandar y se ha usado en otros
papers? Seria bueno mencionar esta para educar al evaluador que no
sabe que esto se hace asi}. \cref{tab:SSClaResults}
shows the results concerning this second experiment. On average, our
CCGPMA-C accomplishes the best predictive AUC; moreover, we note that
CGPMA-C reaches the second-best performance. Furthermore, the
GPs-based competitors achieve competitive results (GPC-MV, MA-GPC,
MA-GPCV, and KAAR). On the other hand, the GPC-MV method obtains a
significantly lower performance than our CCGPMA-C, which is \comment{because
of} its \comment{naive model (lower bound)}{no me queda claro por que
esto explica el menor performance}. Conversely, analyzing the results from
MA-GPC, MA-GPCV, and KAAR, we note that they perform worse than
GPC-MV. We explain such an outcome \comment{in two regards}{in two ways}. First, these
approaches do not model the relationship between the input features
and the annotators' performance. \comment{Second, as presented in a previous
experiment, a \textit{one-vs-all} scheme is adopted to couple
multi-class problems}{no me queda claro por que esta es una de las dos
razones}. The latter can be confirmed in the results for
the multi-class dataset ``Western'' ($K=4$), where the predictive AUC
is lower compared with \comment{the remaining approaches a similar to
GPC-MV}{no entiendo esta sentencia}. Then, analyzing the results from the DL-based strategies, we
note a slightly better performance compared with the GPs-based methods
(excluding CGPMA-C and CCGPMA-C). However, the DL-based performs
considerably worse than our proposal because the CrowdLayer provides
straightforward codification of the labelers' performance to guarantee
a low computational cost~\cite{morales2019scalable1}. Finally, from
the linear models, we first analyze the outstanding performance from
MA-DGRL, which defeats all its non-linear competitors. In particular,
the simulated labels (see \cref{sec:datasets}) follows the MA-DGRL
model, favoring its performance. Though MA-LFC-C achieves competitive
performance compared to the DL-based methods, it is considerably lower
than our proposal. In fact, the MA-LFC-C formulation assumes that the
annotators' behavior is homogeneous across the input space, which does
not correspond to the labels simulation procedure.

\subsection{Fully real data results.}

We test the \emph{fully real datasets}, which configure the most
challenging scenario. The input features and the labels from multiple
experts come from real-world applications. \cref{tab:FSClaResults}
outlines the achieved AUC. First, we observe that for the voice data,
G and R scales exhibit a similar AUC for all considered approaches; in
fact, GPC-MV obtains a result comparable with the upper bound
GPC-GOLD. The latter can be explained in the sense that the annotators
exhibit a suitable performance for these scales, i.e., the provided
labels are similar to the ground truth. On the other hand, a reduction
in the predictive AUC is observed for scale B, which is a consequence
of diminishing the labelers' performance compared with scales G and R,
as demonstrated in \cite{gonzalez2015automatic}. Our approaches
exhibit the best generalization performances for the three scales in
the voice dataset. Remarkably, CGPMA-C and CCGPMA-C do not suffer
significant changes in the scale B, which is an outstanding outcome
because it reflects that our method offers a better representation of
the labelers' behavior against low-quality annotations. Finally, we
review the AUC for the Music dataset. Achieved results show a low
performance for the MA-GPC, even lower than their intuitive lower
bound (GPC-MV). \comment{Remarkably}{la palabra remarkably ya se uso
  en este mismo parrafo. A proposito, no recuerdo que era lo que se
  anotaba en la base de datos de musica, si se explico antes?}, our CCGPMA-C reaches the best predictive
AUC, being comparable with the intuitive upper bound.

\begin{table}[!htb]
	\centering
	%\tiny
	\scriptsize 
	\caption{AUC classification results for the fully real datasets. Bold: the highest performance excluding the GPC-GOLD bound.}
	\resizebox{.98\linewidth}{!}{
	\begin{tabular}{cccccc}\toprule
		\multirow{2}{*}{Method}& \multicolumn{3}{c}{Voice} & \multirow{2}{*}{Music} & \multirow{2}{*}{Average}\\ & G & R & B &&\\\midrule
		GPC-GOLD($M=40$)&$0.9481$ & $0.9481$ & $0.9481$ & $0.9358$ & $0.9450$\\ 
        GPC-GOLD($M=80$)&$0.9484$ & $0.9484$ & $0.9484$ & $0.9178$ & $0.9407$\\ 
        GPC-MV($M=40$)  &$0.8942$ & $0.9373$ & $0.8001$ & $0.8871$ & $0.8797$\\ 
        GPC-MV($M=80$)  &$0.9301$ & $0.9377$ & $0.7962$ & $0.8897$ & $0.8884$\\ 
        MA-LFC-C        &$0.9122$ & $0.9130$ & $0.8406$ & $0.8599$ & $0.8814$\\ 
        MA-DGRL         &$0.9127$ & $0.9164$ & $0.8259$ & $0.8832$ & $0.8845$\\ 
        MA-GPC          &$0.8660$ & $0.8597$ & $0.4489$ & $0.8253$ & $0.7500$\\ 
        MA-GPCV         &$0.9283$ & $0.9208$ & $0.8835$ & $0.8677$ & $0.9001$\\ 
        MA-DL-MW        &$0.8957$ & $0.8966$ & $0.8123$ & $0.8567$ & $0.8653$\\ 
        MA-DL-VW        &$0.8942$ & $0.8929$ & $0.8092$ & $0.9167$ & $0.8782$\\ 
        MA-DL-VW+B      &$0.9030$ & $0.8937$ & $0.8218$ & $0.8573$ & $0.8689$\\ 
        KAAR            &$0.9109$ & $0.9351$ & $0.8969$ & $0.8896$ & $0.9081$\\ 
        CGPMA-C($M=40$) &$\mathbf{0.9324}$ & $0.9406$ & $0.8696$ & $0.9025$ & $0.9113$\\ 
        CGPMA-C($M=80$) &$\mathbf{0.9324}$ & $0.9417$ & $0.8708$ & $0.8987$ & $0.9109$\\ 
        CCGPMA-C($M=40$)&$0.9318$ & $\mathbf{0.9422}$ & $\mathbf{0.9002}$ & $0.9446$ & $\mathbf{0.9297}$\\ 
        CCGPMA-C($M=80$)&$0.9243$ & $0.9383$ & $0.8907$ & $\mathbf{0.9456}$ & $0.9247$\\ \bottomrule
	\end{tabular}}
	\label{tab:FSClaResults}
\end{table}

% \subsection{Regression}



\section{Conclusion}
This paper introduces a novel Gaussian Process-based approach to deal
with Multiple Annotators scenarios, termed Correlated Chain Gaussian
Process for Multiple Annotators (CCGPMA). Our method is built as an
extension of the chained GP~\cite{saul2016chained}, introducing a
semi-parametric latent factor model-(SLFM) to exploit correlations
between the GP latent functions that model the parameters of a given
likelihood function. To the best of our knowledge, CCGPMA is the first
attempt to build a probabilistic framework that codes the annotators'
expertise as a function of the input data and exploits the
correlations among the labelers' answers. Besides, we highlight that
our approach can be used with different likelihood, which allows us to
deal with both categorical data (classification) and real-valued
(regression). For the regression extension see \cref{CCGPMAReg}. We
tested our approach for classification tasks using different scenarios
concerning the provided annotations: synthetic, semi-synthetic,
real-world experts. \comment{Attained}{According} to the results, we remark that our CCGPMA
can achieve robust predictive properties for the studied datasets,
outperforming state-of-the-art methods.

As future work, CCGPMA can be extended by using convolution
processes~\cite{alvarez2011computationally} instead of the SLFM,
aiming to obtain a better representation of the correlations among the
labelers. \comment{On the other hand, our approach can be enhanced towards a
multi-output Gaussian processes framework holding multiple
annotators.}{esto no lo entiendo, pense que esto es lo que estabamos
haciendo en este paper}

% use section* for acknowledgment
\section*{Acknowledgment}
Under grants provided by the Minciencias project: "Desarrollo de un prototipo funcional para el monitoreo no intrusivo de veh\'iculos
usando data analitycs para innnovar en el proceso de mantenimiento basado en la condici\'on en empresas de transporte p\'ublico."-code 643885271399. J. Gil is funded by the program ``Doctorados Nacionales - Convocatoria 785 de 2017''. MAA has been financed by the EPSRC Research Projects EP/R034303/1 and EP/T00343X/1. MAA has also been supported by the Rosetrees Trust (ref: A2501). A.M. Alvarez is financed by the project "Prototipo de interfaz cerebro-computador multimodal para la detección de patrones
relevantes relacionados con trastornos de impulsividad" (Universidad Nacional de Colombia - code 50835).

\bibliographystyle{IEEEtran}
\footnotesize
\bibliography{refs}


% Can use something like this to put references on a page
% by themselves when using endfloat and the captionsoff option.
% \ifCLASSOPTIONcaptionsoff
%   \newpage
% \fi



% trigger a \newpage just before the given reference
% number - used to balance the columns on the last page
% adjust value as needed - may need to be readjusted if
% the document is modified later
%\IEEEtriggeratref{8}
% The "triggered" command can be changed if desired:
%\IEEEtriggercmd{\enlargethispage{-5in}}

% references section

% can use a bibliography generated by BibTeX as a .bbl file
% BibTeX documentation can be easily obtained at:
% http://mirror.ctan.org/biblio/bibtex/contrib/doc/
% The IEEEtran BibTeX style support page is at:
% http://www.michaelshell.org/tex/ieeetran/bibtex/
%\bibliographystyle{IEEEtran}
% argument is your BibTeX string definitions and bibliography database(s)
%\bibliography{IEEEabrv,../bib/paper}
%
% <OR> manually copy in the resultant .bbl file
% set second argument of \begin to the number of references
% (used to reserve space for the reference number labels box)
\appendices
\crefalias{section}{appendix}

\section{CCGPMA applied for regression tasks}\label{CCGPMAReg}
\comment{}{por que esto no se incluye en el material suplementario?}
For real-valued outputs, e.g., $\mathcal{Y} \s{\subset}\Real$, we follow the multi-annotator model used in \cite{raykar2010learning,groot2011learning,xiao2013learning,rodrigues2017learning}, where each output $y_n^r$ is considered to be a corrupted version of the hidden ground truth $y_n$. Then:

\begin{align}
\label{eq:RegLik}
p(\mat{Y}|{\bm{\theta}}) = \prod^N_{n=1}\prod_{r\in R_n}\gauss\left(y_n^r|y_n,v_{n}^r\right),
\end{align}
where $v_{n}^r\en\Real^+$ is the $r$-th annotator error-variance for the instance $n$. In turn, to model this likelihood function with CCGPMA, it is necessary to chain each likelihood's paramater to a latent function $f_j$. Thus, we require $J\igual R+1$ LFs; one to model the hidden ground truth, such that $y_n\igual f_1(\ve{x}_n)$, and $R$ LFs to model each error-variance $v_{n}^r\igual \exp(f_{l_r}(\ve{x}_n))$, with $r\en \left\{1, \dots R\right\}$, and $l_r \igual r+1 \in \left\{2, \dots J\right\}$. Note that we use an exponential function to map from $f_{l_r}$ to $v_{n}^r$, aiming to guarantee $v_{n}^r\s{>}0$ ($f_{l_r}\en \Real$).

\section{ELBO optimization}
\comment{}{por que esto no se incluye en el material suplementario?}
For the sake of clarity, we expose the gradient of $q(\ve{u}_q)$ w.r.t. $\ve{m}_q$, regarding the lower bound in \cref{eq:LowBound21}. For specific details and the remaining gradients, see the supplementary material. Therefore:


\begin{align}
\notag\fracpartial{\mathcal{L}}{\ve{m}_q} \igual& \underbrace{\fracpartial{}{\ve{m}_q}\mathbb{E}_{q(\hat{\ve{f}})}\left[\log p(\mat{Y}|\hat{\ve{f}})\right]}_{\mbox{VE part}} - \cdots\\
& - \cdots \fracpartial{}{\ve{m}_q}\underbrace{\sum_{q=1}^{Q}\mathbb{D}_{KL}(q(\ve{u}_q)||p(\ve{u}_q))}_{\mbox{KL part}},
\end{align}
where the KL part w.r.t. $\ve{m}_q$, yields:

\begin{align}
\fracpartial{}{\ve{m}_q}{\mathbb{D}_{KL}(q(\ve{u}_q)||p(\ve{u}_q))} = \mat{K}_{\ve{u}_q\ve{u}_q}^{-1}\ve{m}_q.
\end{align}

Now, the VE part w.r.t. $\ve{m}_q$ is given as:

\begin{align}
\notag\fracpartial{}{\ve{m}_q}\mathbb{E}_{q(\hat{\ve{f}})}\left[\log p(\mat{Y}|\hat{\ve{f}})\right] &= \fracpartial{}{\ve{m}_q}\mathbb{E}_{q(\hat{\ve{f}})}\left[\log p(\mat{Y}|\hat{\ve{f}})\right],\\
\notag&\quad\; \mbox{Chain Rule:}\; \bm{\mu}_j=g(\ve{m}_q),\\
&= {\mathbb{E}_{q(\hat{\ve{f}})}\left[\fracpartial{}{\hat{\ve{f}}}\log p(\mat{Y}|\hat{\ve{f}})\right]} \fracpartial{\bm{\mu}_j}{\ve{m}_q}.
\end{align}
Following the equality presented in ~\cite{hensman2015scalable}:

\begin{align}
\fracpartial{}{\mu}\mathbb{E}_{\gauss(x|\mu,\sigma^2)}\left[h(x)\right] = \mathbb{E}_{\gauss(x|\mu,\sigma^2)}\left[\fracpartial{}{x}h(x)\right].
\end{align}
Solving $\mathbb{E}_{q(\hat{\ve{f}})}\left[\fracpartial{}{\hat{\ve{f}}}\log p(\mat{Y}|\hat{\ve{f}})\right]$ for the classification likelihood (\cref{eq:ClasLik}):

\begin{align}
		\notag\fracpartial{}{f_{k,n}}\left[\log p(\mat{Y}|\hat{\ve{f}})\right] &=\sum_{r\in R_n}{\lambda}_n^r\left(\delta(y_n^r,k)-\zeta_{k,n}\right),
\end{align}
and

\begin{align}
		\notag\fracpartial{}{f_{l_r,n}}\left[\log p(\mat{Y}|\hat{\ve{f}})\right]	& = \begin{cases}
		\Omega, & \mbox{if $r\en R_n$}\\
		0, & \mbox{Otherwise}
		\end{cases},
\end{align}
where

\begin{align}
		\notag\Omega &=\left({\lambda}_n^r - ({\lambda}_n^r)^2\right)\left(\sum_{k=1}^{K}\delta(y_n^r,k)\log\left(\zeta_{k,n}\right) + \log(K)\right).
\end{align}

Moreover,
\begin{align}
\label{eq:Expfkn}
	\notag \mathbb{E}_{q(\hat{\ve{f}})}\left[\fracpartial{}{f_{k,n}}\left[\log p(\mat{Y}|\hat{\ve{f}})\right]\right] &= \sum_{r\in R_n}\mathbb{E}_{q(f_{l_r,n})}\left[{\lambda}_n^r\right]\times \cdots\\
	&\hspace{-2.0cm} \cdots \times  \left(\delta(y_n^r,k) - \mathbb{E}_{q(f_{1,n})\dots q(f_{K,n})}[\zeta_{k,n}]\right),
\end{align}
and 

\begin{align}
    \notag \mathbb{E}_{q(\hat{\ve{f}})}\left[\fracpartial{}{f_{l_r,n}}\left[\log p(\mat{Y}|\hat{\ve{f}})\right]\right] &=\\
    &\hspace{-2.5cm}=
    \begin{cases}
    \mathbb{E}_{q(f_{1,n})\dots q(f_{K,n})q(f_{l_r,n})}\left[\Omega\right], & \mbox{if } r\en R_n\\
    0, & \mbox{Otherwise}
    \end{cases},
\end{align}
where

\begin{align}
    \label{eq:Expflr}
	\notag\mathbb{E}_{q(f_{1,n})\dots q(f_{K,n})q(f_{l_r,n})}\left[\Omega\right] &\igual\\
	&\notag\hspace{-3.6cm}\igual\left(\mathbb{E}_{q(f_{l_r,n})}[{\lambda}_n^r]-\mathbb{E}_{q(f_{l_r,n})}[({\lambda}_n^r)^2]\right)\times\cdots\\
	&\hspace{-3.6cm}\cdots\times\left(\sum_{k=1}^{K}\delta(y_n^r,k)\mathbb{E}_{q(f_{1,n})\dots q(f_{K,n})}\left[\log\left(\zeta_{k,n}\right)\right] + \log(K) \right).
\end{align}
Notice that the expected values in \cref{eq:Expfkn} and \cref{eq:Expflr} have no analytical solution. Therefore, we approximate them by using the Gauss-Hermite approach~\cite{hensman2015scalable,saul2016chained}.
		

% \subsection{Datasets and simulated/provided annotations}\label{dataReg}
% To control the label generation~\cite{ruiz2019learning}, we build \textit{semi-synthetic data} from six datasets related to regression tasks from the well-known {UCI repository}. We selected the following datasets: {Auto MPG Data Set}--(Auto), {Bike Sharing Dataset Data Set}--(Bike), {Concrete Compressive Strength Data Set}--(Concrete), {The Boston Housing Dataset}--(Housing),\footnote{See https://www.cs.toronto.edu/$\sim${d}elve/data/boston/bostonDetail.html for housing} {Yacht Hydrodynamics Data Set}--(Yacht), and {Relative location of CT slices on axial axis Data Set}--(CT).  
% Third, we evaluate our proposal on one \textit{fully real dataset}, we use the Music dataset introduced in0\cref{sec:datasets}. Notice that the music dataset configures a 10-class classification problem; however in this first experiment we are using our CCGPMA with a likelihood function designed for real-valued labels \cref{eq:RegLik}. Such practice is not uncommon in machine learning, and it is usually known as ``Least-square classification'' \cite{rasmussen2006gaussian}. \cref{tab:RegData} summarizes the tested datasets for the regression case.
% \begin{table}[!tb]
% 	\caption{Datasets for regression.
% 	}
% 	\label{tab:RegData}
% 	\centering
% 	\resizebox{\linewidth}{!}{
% 		\begin{tabular}{ccccC{2cm}cC{2cm}}\toprule
% 			&& Name && Number of features && Number of instances\\\midrule
% 		\multirow{ 1}{*}{\textit{fully synthetic}}	&& synthetic && 1 && 100\\\midrule
% 		\multirow{ 6}{*}{\textit{semi-synthetic}}   && Auto&& 8 && 398\\ 
% 		                                            && Bike && 13 && 17389\\
% 		                                            && Concrete && 9 && 1030\\
% 		                                            && Housing && 13 && 506\\
% 		                                            && Yacht && 6 && 308\\
% 		                                            && CT && 384&&53500
% 		                                            \\\midrule
%       	\multirow{ 1}{*}{\textit{fully real}}       && Music && 124 && 1000
%       	\\\bottomrule
% 	\end{tabular}}
% \end{table}
% As we pointed out previously, \textit{fully synthetic} and \textit{semi-synthetic} datasets do not hold real annotations. Thus, it is necessary to generate these labels synthetically as a version of the gold standard corrupted by Gaussian noise, i.e., $y_n^r = y_n +\epsilon^r_{n}$, where $\epsilon^r_{n}\sim \gauss(0, v^r_{n})$, being $v^r_{n}$ the $r$-th annotator error-variance for the sample $n$. Note that we are interested in interested in modelling such an error-variance for the $r$-th annotator as a function of the input features, which is correlated with the variances of the other labelers. For doing so, the error variances are generated as follows:
% \begin{itemize}
%     \item Define $Q$ functions $\mu_q(\cdot)$, and the combination parameters $w_{l_r,q},\,\forall r, q$.
%     \item For each annotator $r$ and sample $n$, compute $f_{l_r,n} = \sum_{q=1}^{Q}w_{l_r,q}\mu_q(\hat{x}_n)$, where $\hat{x}_n$ is the $n$-th component of $\hat{\ve{x}}\in \Real$, which is an $1-$D representation of input features $\mat{X}$ by using the t-distributed Stochastic Neighbor Embedding approach \cite{maaten2008visualizing}.
%     \item Finally, determine $v^r_n = \exp(f_{l_r,n})$. 
% \end{itemize}
% \subsection{Method comparison and performance metrics}
% \begin{table}[bt!]
% 	\caption{A brief overview of state-of-the-art methods tested for regression tasks. GPR: Gaussian Processes Regression, LR: logistic regression, Av: average, MA: multiple annotators, DL: Deep learning, LFCR: Learning from crowds for regression.
% 	}
% 	\label{tab:RegVal}
% 	\centering
% 	\resizebox{1\linewidth}{!}{
% 		\begin{tabular}{lcl}\toprule
% 			Algorithm && Description \\\midrule
% 			GPR-GOLD  && A GPR using the real labels (upper bound).\\
% 			GPR-Av    &&  A GPR using the average of the labels as the ground truth.\\
% 			MA-LFCR~\cite{raykar2010learning}  && A LR model for MA where the labelers' parameters\\
% 			&& are supposed to be constant across the input space.\\
% 			%MA-LMO~\cite{xiao2013learning}   && A multi-labeler approach where\\
% 			%&&the sources parameters depend on the input space.\\
% 			MA-GPR~\cite{rodrigues2014gaussian}	 &&  A multi-labeler GPR, which is as an extension of MA-LFCR.\\
% 			MA-DL~\cite{rodrigues2018deep}  && A Crowd Layer for DL, where the annotators' parameters\\
% 			&&  are constant across the input space.\\
% 			CGPMA-R  &&  A particular case of our CCGPMA for regression,\\
%     		&& where $Q=J$, and $w_{j,q}\igual 1$ if $j\igual q,$ otherwise $w_{j,q}\igual 0.$\\\bottomrule
% 	\end{tabular}}
% \end{table}
% The quality assessment is carried out by estimating the regression performance as the coefficient of determination--($R^2$). A cross-validation scheme is employed with 15 repetitions where $70$\% of the samples are utilized for training and the remaining $30\%$ for testing (except for \textit{fully synthetic dataset}, since it clearly defines the training and testing sets). \cref{tab:RegVal} displays the employed methods of the state-of-the-art for comparison purposes. From \cref{tab:RegVal}, we highlight that for the model MA-DL, the authors provided three different annotators' codification: MA-DL-B, where the bias for the annotators is measured; MA-DL-S, where the labelers' scale is computed; and measured; MA-DL-B+S, which is a version with both \cite{rodrigues2018deep}. 
% \subsection{Results over semi-synthetic data}
% We perform a controlled experiment aiming to verify the capability of our CCGPMA to estimate the performance of inconsistent annotators as a function of the input space and taking into account their dependencies. For this first experiment, we use the \textit{semi-synthetic} dataset described in \cref{dataReg}. We simulate five labelers ($R=5$) with different levels of expertise. To simulate the error-variances, we define $Q=3$ functions $\mu_q(\cdot)$, which are given as 
% \begin{align}
% \label{eq:u1r}
% \notag \mu_1(x) &= 4.5\cos(2\pi x + 1.5\pi) - 3\sin(4.3\pi x + 0.3\pi) +\cdots\\ & \cdots + 4\cos(7\pi x + 2.4\pi),\\
% \label{eq:u2r}
% \notag \mu_2(x) &= 4.5\cos(1.5\pi x + 0.5\pi) + 5\sin(3\pi x + 1.5\pi) - \cdots\\ & \cdots - 4.5\cos(8\pi x+ 0.25\pi),\\
% \label{eq:u3r}
% \mu_3(x) &= 1,
% \end{align}
% where $x\in [0,1]$. Besides, we define the following combination matrix $\mat{W} \in \Real^{Q\times R}$, where
% \begin{align}
% \mat{W}=\begin{bmatrix}
% -0.10  &  0.01   & -0.05 &  0.01  & -0.01\\
% 0.10   &  -0.01  & 0.01  &  -0.05 & 0.05\\
% -2.3   &  -1.77  & 0.54  &  0.9   & 1.42
% \end{bmatrix},
% \label{eq:parametersP}
% \end{align}
% holding elements $w_{l_r,q}$. 
% \begin{table*}[!b]
% 	\centering
% 	%\tiny
% 	\scriptsize 
% 	\caption{Regression results in terms of $R^2$ score over \textit{semi synthetic datasets}. Bold: the highest $R^2$ excluding the upper bound GPR-GOLD.}
% 	\resizebox{.98\linewidth}{!}{
% 	\begin{tabular}{cccccccc}\toprule
% 		{Method} & {Auto} & {Bike} & {Concrete} & {Housing} & {Yacht} & {CT} & {Average}\\\midrule
% 		GPR-GOLD($M=40$)&$0.8604\pm0.0271$ & $0.5529\pm0.0065$ & $0.8037\pm0.0254$ & $0.8235\pm0.0419$ & $0.8354\pm0.0412$ & $0.8569\pm0.0055$ & $0.7888$\\ 
%         GPR-GOLD($M=80$)&$0.8612\pm0.0279$ & $0.5603\pm0.0063$ & $0.8271\pm0.0230$ & $0.8275\pm0.0399$ & $0.8087\pm0.0423$ & $0.8648\pm0.0047$ & $0.7916$\\ 
%         GPR-Av($M=40$)  &$0.8425\pm0.0286$ & $0.5280\pm0.0100$ & $0.7589\pm0.0279$ & $0.7834\pm0.0463$ & $0.7588\pm0.0498$ & $0.8070\pm0.0130$ & $0.7464$\\ 
%         GPR-Av($M=80$)  &$0.8406\pm0.0304$ & $0.5397\pm0.0085$ & $0.7765\pm0.0274$ & $0.7903\pm0.0451$ & $0.7676\pm0.0535$ & $0.8167\pm0.0089$ & $0.7552$\\ 
%         MA-LFCR         &$0.7973\pm0.0218$ & $0.3385\pm0.0051$ & $0.6064\pm0.0384$ & $0.7122\pm0.0509$ & $0.6403\pm0.0186$ & $\mathbf{0.8400\pm0.0014}$ & $0.6558$\\ 
%         MA-GPR          &$0.8456\pm0.0281$ & $0.4448\pm0.0187$ & $0.7769\pm0.0367$ & $0.7685\pm0.0632$ & $0.7842\pm0.1027$ & $0.0105\pm0.0045$ & $0.6051$\\
%         MA-DL-B         &$0.7766\pm0.0253$ & $0.5854\pm0.0107$ & $0.2319\pm0.0328$ & $0.5317\pm0.1005$ & $0.2089\pm0.0783$ & $0.6903\pm0.2689$ & $0.5041$\\ 
%         MA-DL-S         &$0.7761\pm0.0279$ & $\mathbf{0.5828\pm0.0149}$ & $0.2363\pm0.0252$ & $0.5352\pm0.0948$ & $0.1822\pm0.0985$ & $0.9394\pm0.0257$ & $0.5420$\\ 
%         MA-DL-B+S       &$0.7717\pm0.0239$ & $0.5816\pm0.0181$ & $0.2369\pm0.0322$ & $0.5330\pm0.0850$ & $0.1974\pm0.0895$ & $0.5517\pm0.2316$ & $0.4787$\\ 
%         CGPMA-R($M=40$) &$0.8474\pm0.0221$ & $0.5464\pm0.0069$ & $0.8169\pm0.0231$ & $0.7946\pm0.0498$ & $0.7545\pm0.1029$ & $0.8236\pm0.0132$ & $0.7639$\\ 
%         CGPMA-R($M=80$) &$0.7768\pm0.0708$ & $0.5560\pm0.0074$ & $0.8190\pm0.0254$ & $0.8058\pm0.0493$ & $0.8230\pm0.0760$ & $0.8371\pm0.0104$ & $0.7696$\\ 
%         CCGPMA-R($M=40$)&$0.8563\pm0.0247$ & $0.5284\pm0.0117$ & $0.7976\pm0.0270$ & $0.7994\pm0.0462$ & $0.8436\pm0.0507$ & $0.8219\pm0.0062$ & $0.7745$\\ 
%         CCGPMA-R($M=80$)&$\mathbf{0.8578\pm0.0244}$ & $0.5467\pm0.0069$ & $\mathbf{0.8220\pm0.0259}$ & $\mathbf{0.8110\pm0.0453}$ & $\mathbf{0.8476\pm0.0544}$ & $0.8252\pm0.0083$ & $\mathbf{0.7850}$\\\bottomrule
% 	\end{tabular}}
% 	\label{tab:SSRegResults}
% \end{table*}
% \cref{tab:SSRegResults} shows the results the experiment with \textit{semi synthetic dataset}. On average, our CCGPMA-R  exhibits the best generalization performance in terms of the $R^2$ score. On the other hand, regarding its GPs-based competitors (GPR-Av, MA-GPR, and CGPMA-R), we first note that the performance of CGPMA-R exhibit a similar (but lower) performance than CCGPMA-R. The above is a consequence of that conversely to CGPMA-R our CCGPMA-R model the annotators' interdependencies. Secondly, the intuitive lower bound GPR-Av exhibits a significantly worse prediction than our approaches. On the other hand, we remark the behavior of MA-GPR, which is lowest compared with its GPs-based competitors, even far worse than the supposed lower bound GPR-Av. The key to this abnormal outcome lies in the formulation of this approach; MA-GPR models the annotators' behavior by assuming that their performance does not depend on the input features and considering that the labelers make their decisions independently, which does fit the process that we use to simulate the labels for this experiment.
% Next, we analyze the results concerning the linear model MA-LFR; attained to the results, we note that this approach's prediction capacity is far lower than our approaches; the above outcome suggests that there may exist a non-linear structure in most databases. However, we highlight a particular result for the dataset CT, where MA-LFCR exhibits the best performance defeating all its competitors based on non-linear models. From the above, we intuit that the CT dataset may have a linear structure. To confirm this supposition, we perform an additional experiment over CT by training a regression scheme based on LR with the actual labels (we follow the same scheme as for GPR-GOLD). We obtain an $R^2$ score equal to $0.8541$ (on average), which is close to the results obtained by GPR-GOLD. Thus, we can elucidate that there exists a linear structure in the dataset CT. Finally, we analyze the results for the DL-based models. Similar to the experiments over \textit{fully synthetic datasets}, we note a considerable low prediction capacity; in fact, they are even defeated by the linear model MA-LFR. Again, we attribute this behavior to the fact that the CrowdLayer (used to manage the data from multiple annotators) does not offer a suitable codification of the labelers' behavior. Nevertheless, taking the above into account, we observe an unusual result in the dataset Bike, where the DL-based approaches offer the best performance, even defeating the supposed upper-bound GPR-GOLD. To explain that, it is necessary to analyze the meaning of the target variable in such a dataset. Regarding to the description of this dataset,\footnote{Such description can be found in https://archive.ics.uci.edu/ml/datasets/bike+sharing+dataset} the target variables indicate the count of total rental bikes, including both casual and registered in a day. The above suggests that there may exist a quasi-periodic structure in the dataset, which cannot be captured by the GPR-GOLD since it uses a non-periodic kernel (it uses the RBF kernel). To support our suppositions, an additional experiment was performed over this dataset by training the model GPR-GOLD with the kernel defined as follows. 
% \begin{align}\label{eq:Pkernel}
% \kappa(\ve{x}_n, \ve{x}_{n^{\prime}}) = \varphi \exp \left[  - \frac{1}{2}\sum_{p=1}^{P}\left( \frac{\sin(\frac{\pi}{T_p} (x_{p,n}- x_{p,n^{\prime}}) )}{l_p}\right)^2 \right],
% \end{align}
% where $\varphi\in \Real$ is the variance parameter, $l_p\in (\Real^{+})$ is the length-scale parameter for the $p$-th dimension, and $T_p\in (\Real^{+})$ is the period for the $p$-th dimension. Therefore, we obtain an $R^2$ score equal to $0.5952$ (on average), which is greater than the obtained by the DL-based approaches, indicating a quasi-periodic structure in the Bike dataset as we had supposed.
% biography section
% 
% If you have an EPS/PDF photo (graphicx package needed) extra braces are
% needed around the contents of the optional argument to biography to prevent
% the LaTeX parser from getting confused when it sees the complicated
% \includegraphics command within an optional argument. (You could create
% your own custom macro containing the \includegraphics command to make things
% simpler here.)
%\begin{IEEEbiography}[{\includegraphics[width=1in,height=1.25in,clip,keepaspectratio]{mshell}}]{Michael Shell}
% or if you just want to reserve a space for a photo:

% \begin{IEEEbiography}{Michael Shell}
% Biography text here.
% \end{IEEEbiography}
% % if you will not have a photo at all:


% insert where needed to balance the two columns on the last page with
% biographies
%\newpage

% \begin{IEEEbiographynophoto}{Jane Doe}
% Biography text here.
% \end{IEEEbiographynophoto}

% You can push biographies down or up by placing
% a \vfill before or after them. The appropriate
% use of \vfill depends on what kind of text is
% on the last page and whether or not the columns
% are being equalized.

%\vfill

% Can be used to pull up biographies so that the bottom of the last one
% is flush with the other column.
%\enlargethispage{-5in}



% that's all folks

\vspace{-1.0cm}
\begin{IEEEbiographynophoto}{J. Gil-Gonzalez}
received his undergraduate degree in electronic engineering (2014) from the Universidad Tecnoloǵica de Pereira, Colombia. His M.Sc. in electrical engineering (2016) from the same university. Currently, he is PhD student from the same university. His research interests include probabilistic models for machine learning, learning from crowds, and Bayesian inference.
\end{IEEEbiographynophoto}
\vspace{-1.0cm}
\begin{IEEEbiographynophoto}{Juan-José Giraldo}
Received a degree in Electronics Engineering (B. Eng.)
with Honours, from Universidad del Quind\'io, Colombia in 2009, a master degree in Electrical Engineering (M. Eng.) from Universidad Tecnológica de Pereira, Colombia in 2015. Currently, Mr. Giraldo is a Ph.D student in Comp. Science at the University of Sheffield, UK
\end{IEEEbiographynophoto}
\vspace{-1.0cm}
\begin{IEEEbiographynophoto}{A.M. Álvarez-Meza}
received his undergraduate degree in electronic engineering (2009), his M.Sc. degree in engineering (2011), and his Ph.D. in automatics from the Universidad Nacional de Colombia. He is a Professor in the Department of Electrical, Electronic and Computation Engineering at the Universidad Nacional de Colombia - Manizales. His research interests include machine learning and signal processing.
\end{IEEEbiographynophoto}
\vspace{-1.0cm}
\begin{IEEEbiographynophoto}{A. Orozco-Gutierrez}
received his undergraduate degree in electrical engineering (1985) and his M.Sc. degree in electrical engineering (2004) from Universidad Tecnoloǵica de Pereira, and his Ph.D. in bioengineering (2009) from Universidad Politecnica de Valencia (Spain). He received his undergraduate degree in law (1996) from Universidad Libre de Colombia. He is a Professor in the Department of Electrical Engineering at the Universidad Tecnologica de Pereira. His research interests include bioengineering.
\end{IEEEbiographynophoto}
\vspace{-1.0cm}
\begin{IEEEbiographynophoto}{M. A. Álvarez}
received the BEng degree in electronics engineering from the Universidad Nacional de Colombia (2004), the M.Sc. degree
in electrical engineering from the Universidad Tecnologica de Pereira, Colombia (2006), and the PhD degree in computer science from The University of Manchester, UK (2011). Currently, he is a Lecturer of Machine Learning in the Department of Computer Science, University of Sheffield, United Kingdom. His research interests include probabilistic models, kernel methods, and stochastic processes.
\end{IEEEbiographynophoto}


\end{document}


